%%%%%%%%%%%%%%%%%%%%%%%%%%%%%%%%%%%%%%%%%%%%%%%%%%%%%%%%%%%%%%%%%%%%%%%%%%%%%%%%
% predictions.tex: Selected events predictions at Far Detector
%%%%%%%%%%%%%%%%%%%%%%%%%%%%%%%%%%%%%%%%%%%%%%%%%%%%%%%%%%%%%%%%%%%%%%%%%%%%%%%%
\chapter{Far Detector Prediction and Analysis}
\label{prediction_chapter}
%%%%%%%%%%%%%%%%%%%%%%%%%%%%%%%%%%%%%%%%%%%%%%%%%%%%%%%%%%%%%%%%%%%%%%%%%%%%%%%%

As every high energy physics experiment $\nu_\mu$ disappearance analysis, first, requires to make
a selected samples prediction for specific values of oscillation parameters $\Delta m^2_{32}$ and 
$\theta_{23}$ and second, compares these predicted samples with a real selected samples. A measurement 
of parameters is obtained from a method known as a likelihood fit. The simplest way to do it is to 
compare directly FD prediction with data, however, a significant flux and neutrino cross section
uncertainties would result in not so great experiment sensitivity. The existence of ND and its measurements
of neutrino spectum mitigate greatly aforementioned problem. A procedure, called extrapolation, 
helps to predict FD neutrino spectrum based on ND measurements and is described in the current chapter.

\section{Prediction}
Predicted FD spectrum consists of $\nu_\mu$ CC signal, beam background components and cosmic background.
Extrapolation procedure uses ND data only to predict $\nu_\mu$ CC signal since it is a significant
portion of the FD neutrino spectrum. Beam background components are taken directly from FD Monte Carlo
and cosmic background is estimated based on data recorded outside the beam spill window of NuMI spill. 

\subsection{Signal Extrapolation} \label{extrap_procedure}
Monte Carlo simulations for ND and FD are genereated separately and under assumption of no neutrino
oscillation. The shape of reconstructed neutrino spetra at different detectors is different largely due 
to the fact of different geometric effects and inconsistency in distortions caused by different 
resolutions of energy estimators at ND and FD. Figure \ref{fig:ND_FD_shapes} illustrates the overlay
of neutrino spectra shapes for both detectors.

Being a small detector ND cannot contain a highly energetic neutrinos, thus its observed neutrino
spectrum is based towards lower energy - smaller hadronic activite and shorter muon tracks. However,
steal plates alternating with regular detectors planes at the end of ND help to contain more
energetic muons, nevertheless neutrino spectrum is limited by around 4 GeV. One more geometrical effect
which plays a role in FD/ND spectra shape difference is that decay tube\footnote{Place where pions
decay into neutrinos and other particles} is not seen as a point neutrino source. ND catch neutrinos
which was produced at wider range of off-axis angle and therefore ND sees broader neutrino spectrum 
as compared to FD. All these effects are well model by NOvA simulations.

Distortions caused by different resolutions of energy estimators at ND and FD is also metigated by
extrapolation procedure. For every detector a 2D histogram of reconstructed energy vs. true energy
is created. These \textit{reco-true matricies} is used to convert bins of reconstructed energy at
each detector to bins of true energy. 

The full extrapolation procedure works as follows. Firstly, an observed ND neutrino spectum in bins of 
reconstructed energy is converted to the spectrum in bins of true energy with the help reco-true 
matrix. Secondly, geometric effect is metigated by multiplying result of previous step with FD/ND 
true energy ratio to get FD unoscillated prediction in bins of true energy. Thirdly, every bin of the 
spectrum is reweighted with oscillated probability. Finally, FD reco-true matrix is applied to 
convert neutrino spectrum in bin of true energy back to spectrum in bins of reconstructed energy,
thus result is the predicted FD neutrino spectrum based on the observed neutrinos at the ND. 
Figure \ref{fig:extrap_scheme} illustrates steps of the extrapolation procedure.
\begin{figure}[!th]
\centering
\includegraphics[width=1.0\textwidth]{figures/extrapolation_scheme.pdf}
\caption{Extrapolation procedure schematic.}
{To obtain a correct prediction of neutrino spectrum at the FD based on the observed neutrino 
spectrum at the ND 4 steps are needed to take. Convert ND spectrum in bins of reconstructed energy to
spectrum in bins of true energy by reco-true matrix. Multiply by FD/ND true energy ratio to account
for a different geometric effects for the FD and the ND. Reweight every bin of the spectrum by
oscillation probability. Convert spectrum in bins of true energy back to spectrum in bins of
recontructed energy. }
\label{fig:extrap_scheme}
\end{figure}

\subsection{Beam Background Prediction}
As mentioned in the begginig of the section, only $\nu_\mu$ CC signal\footnote{Signal consists of two
channels: $\nu_\mu \rightarrow \nu_\mu$ and $\bar\nu_\mu \rightarrow \bar\nu_\mu$.} is passing 
through extrapolation procedure described in section \ref{extrap_procedure}. All beam induced 
background components, namely $\nu_e, \bar\nu_e, \nu_\tau, \bar\nu_\tau$ and NC, are small compare 
to signal and ND Monte Carlo is substracted from observed neutrino spectrum before it get extrapolated.
Then beam background from FD Monte Carlo is added back to predicted FD neutrino spectrum \cite{extrap_technote}. 

\subsection{Cosmic Background Prediction}
Since for every 10 $\mu$s of NuMI spill a wider window of 450 $\mu$s is recorded the same data files
are used for the cosmic background estimation. All selected events outside of NuMI spill window are 
normalized by scale factor 
\be
\text{scale} = \frac{10 \mu\text{s}}{450 \mu\text{s} - 10 \mu\text{s}},
\ee
to estimate cosmic background inside the NuMI spill window. After than, this selected spectrum is added 
to the FD predicted neutrino spectrum.

\section{Analysis}
This section describes a procedure used for the measurement of oscillation parameters $\Delta m^2_{32}$ 
and $\sin^2\theta_{23}$. The procedure is called a binned maximal-likelihood fitting.

\subsection{4+1 Sample Fitting}
ND neutrino spectrum at some point reweighted with oscillation probability during the extrapolation procedure 
%%%%%%%%%%%%%%%%%%%%%%%%%%%%%%%%%%%%%%%%%%%%%%%%%%%%%%%%%%%%%%%%%%%%%%%%%%%%%%%%
