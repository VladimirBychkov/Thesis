%%%%%%%%%%%%%%%%%%%%%%%%%%%%%%%%%%%%%%%%%%%%%%%%%%%%%%%%%%%%%%%%%%%%%%%%%%%%%%%%
% predictions.tex: Selected events predictions at Far Detector
%%%%%%%%%%%%%%%%%%%%%%%%%%%%%%%%%%%%%%%%%%%%%%%%%%%%%%%%%%%%%%%%%%%%%%%%%%%%%%%%
\chapter{Far Detector Prediction and Analysis}
\label{prediction_chapter}
%%%%%%%%%%%%%%%%%%%%%%%%%%%%%%%%%%%%%%%%%%%%%%%%%%%%%%%%%%%%%%%%%%%%%%%%%%%%%%%%

As every high energy physics experiment $\nu_\mu$ disappearance analysis, first, requires to make
a selected samples prediction for specific values of oscillation parameters $\Delta m^2_{32}$ and 
$\theta_{23}$ and second, compares these predicted samples with a real selected samples. A measurement 
of parameters is obtained from a method known as a likelihood fit. The simplest way to do it is to 
compare directly FD prediction with data, however, a significant flux and neutrino cross section
uncertainties would result in not so great experiment sensitivity. The existence of ND and its measurements
of neutrino spectum mitigate greatly aforementioned problem. A procedure, called extrapolation, 
helps to predict FD neutrino spectrum based on ND measurements and is described in the current chapter.

\section{Prediction}
Predicted FD spectrum consists of $\nu_\mu$ CC signal, beam background components and cosmic background.
Extrapolation procedure uses ND data only to predict $\nu_\mu$ CC signal since it is a significant
portion of the FD neutrino spectra. Beam background components are taken directly from FD Monte Carlo
and cosmic background is estimated based on data recorded outside the beam spill window of NuMI spill. 

\subsection{Signal Extrapolation}

%%%%%%%%%%%%%%%%%%%%%%%%%%%%%%%%%%%%%%%%%%%%%%%%%%%%%%%%%%%%%%%%%%%%%%%%%%%%%%%%
