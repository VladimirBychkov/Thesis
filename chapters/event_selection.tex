%%%%%%%%%%%%%%%%%%%%%%%%%%%%%%%%%%%%%%%%%%%%%%%%%%%%%%%%%%%%%%%%%%%%%%%%%%%%%%%%
% event_selection.tex: Select of showering and tracking events:
%%%%%%%%%%%%%%%%%%%%%%%%%%%%%%%%%%%%%%%%%%%%%%%%%%%%%%%%%%%%%%%%%%%%%%%%%%%%%%%%
\chapter{$\nu_\mu$ CC Selection}
\label{event_selection_chapter}
%%%%%%%%%%%%%%%%%%%%%%%%%%%%%%%%%%%%%%%%%%%%%%%%%%%%%%%%%%%%%%%%%%%%%%%%%%%%%%%%

This and the next chapter constitute the main part of current thesis. As was stated before
all additional information which could be extracted from data improves current measurements 
of $\theta_{23}$ and $\Delta m_{32}^2$. In this chapter, 


\section{Contained Sample}
Contained sample is selected with the help of series cuts which ensure that passed events
are $\nu_\mu$ CC events with good energy resolution and low cosmic contamination. Several
next subsecitions are briefly describes these cuts.

\subsection{Data Quality}
Data quality is the first cut in contained sample selection flow and serves to remove
events of poor quality. Events which pass data quality cut have
\begin{itemize}
\item more than 20 reconstructed hits. There is no useful inforamtion for events with fewer 
number of hits.
\item at least one track reconstructed by cosmic tracker. This criterion confirms that there is
cluster of hits which could be associated with track.
\item PID value of ReMId selector should be greater than zero. This ensures that at least on 
3D Kalman track was reconstructed.
\item slice which spans more than 4 cosecutive planes. This cut removes events with geometric 
gaps and highly vertical tracks, thus the cut is a good preselector for a good fraction of 
cosmic rays.
\end{itemize}

\subsection{Containment} \label{Containment}
The main purpose of containment cut is to select events with activity happening inside the 
detector. This allows partially to remove cosmic rays since they enter detector from the top 
or the side edges. Contained events have the best energy resolution and satisfy following 
criteria
\begin{itemize}
\item Kalman Forward Cell $>$ 6,
\item Kalman Backward Cell $>$ 6,
\item Cosmic Forward Cell $>$ 0,
\item Cosmic Backward Cell $>$ 7,
\item Planes to Front $>$ 1,
\item Planes to Back $>$ 1,
\item Minimal Prong Distance to FD Top $>$ 60,
\item Minimal Prong Distance to FD Bottom $>$ 12,
\item Minimal Prong Distance to FD East $>$ 16,
\item Minimal Prong Distance to FD West $>$ 12,
\item Minimal Prong Distance to FD Front $>$ 18,
\item Minimal Prong Distance to FD Back $>$ 18.
\end{itemize}
First four criteria use tracks reconstructed with the help of Kalman Tracker and Cosmic Tracker.
For every track, number of projected cells\footnote{along the track} from start/end of the track 
to the edge of the detector is estimated and these numbers should be bigger than above stated
tresholds. The next two criteria checks that reconstructed slice does not have hits in the first
and the last planes of detector active region\footnote{Data were taken during detector construction
period, thus active region was changing with time.}. Finally, the last six criteria use objects
reconstructed by Fuzzy clustering algorithm, the objects consist of hits that could belong to 
different clusters (in other words, two particle which move in close trajectories could contribute
to one hit). The tresholds in the last six criteria are result of optimization carried out in 
\cite{numucont_technote}.

\subsection{$\nu_\mu$ CC selection}
This cut aims to select $\nu_\mu$ CC like events and to reject $\nu_e$ CC and $\nu$ NC like events.
Extensive optimization analysis is done in \cite{numupid_technote} and result are folowong
\begin{itemize}
\item ReMId $>$ 0.5 - event has long muon track with a good ReMId score,
\item CVN $>$ 0.5 - event has a good score for $\nu_\mu$ CC classifier based on convolutional neural net.
\end{itemize}

\subsection{Cosmic Rejection}
The final cut is cosmic rejection and its purpose is to remove cosmic events which might slipped in after
previous cuts. Several variables are extracted from the event and input into Boosted Decision Tree (BDT)
algorithm\footnote{See next chapter for more information about BDT algorithm.}\cite{cosrej_technote}. 
These features are
\begin{itemize}
\item cosine of angle between muon track and incoming neutrino,
\item reconstructed y-deriction of the muon track,
\item reconstructed length of the muon track,
\item largest y-position of either the start or the end of muon the track,
\item CVN cosmic score,
\item min(CosmicForwardCell + CosmicBackwardCell, KalmanForwardCell + KalmanBackwardCell), this 
variable helps to descriminate going-through cosmic muons which passed all previous cuts,
\item ratio of hit number in kalman track to hit number in slice, this ratio is a measure of hadronic
energy fraction. Cosmic events have very little hadronic energy as compared to beam neutrino events. 
\end{itemize}
BDT algorithm outputs a PID value and the cut choice is based on optimization carried out in 
\cite{numupid_technote}. Events pass cosmic rejection selection if PID value is grater than $0.5$.

\subsection{Resolution Bins}
Previous analysis carried out in \cite{Kanika}, \cite{Dominick}, \cite{Susan} used one selected sample for 
fitting to actual data. Analysis carried out in \cite{Nick} used several samples for fitting to data 
such as contained QE sample, contained non-QE sample and escaping sample\footnote{Most of the events 
selected in escaping sample in \cite{Nick} have energy far from first oscillation dip, thus it is not 
contributing much to the disappearance analysis. Current analysis aims to address this issue.}. This 
approach separates events into different energy resolution bins and there by improves final sensitivity 
of the experiment. Finally, recent analysis \cite{Luke} goes further and separates contained sample into 
four independent ones with different fraction of hadronic energy in selected events. 

Hadronic energy fraction ($\frac{E_{had}}{E_\nu}$) distribution in made for every reconstructed energy
bin and than it splitted into four equal parts - lowest, second to lowest, second to highest and highest
quantiles - so every quantile has equal number of events. Plot of $\frac{E_{had}}{E_\nu}$ vs $E_\nu$ and 
quantile bondaries are shown in \ref{XXX}. Since detector configuration was different for different run 
periods, the analysis uses different quantiles boundaries to account for that. 

\section{Escaping Sample}
The rest of the chapter is an original work devoted to selection of escaping sample with minimal
cosmic backrgound and maximal beam neutrino signal.

\subsection{Definition And Basic Selection}
Escaping sample does not have a nice quality of contained sample such as low cosmic background and
high energy resolution nevertheless it might provide additional information on neutrino oscillation
effectively increasing gathered statistics. 

Events which define escaping sample do not need to be consist of particles which leave the detector but
rather ones which fail to pass some cuts of contained sample. The obvious cut escaping events should 
fail is containment cut \ref{Containment}. Escaping events also should pass some cuts from contained 
sample because it is still needed that they contain useful information for the analysis. Those cuts are
\begin{itemize}
\item data quality cut - events have reasonable number of reconstructed hits and tracks,
\item CVN $>$ 0.5 - events are highly $\nu_\mu$ CC events,
\item loose cosmic rejection, cosmic rejection PID value is $>$ 0.48 - events should not to be an obvious 
cosmic rays.
\end{itemize}

Selections above are the basic criteria for the escaping sample. Figure \ref{YYY} illustrates expected 
number of cosmic and signal events which pass basic criteria for primary track length distribution.

\subsection{Preselection}

%%%%%%%%%%%%%%%%%%%%%%%%%%%%%%%%%%%%%%%%%%%%%%%%%%%%%%%%%%%%%%%%%%%%%%%%%%%%%%%%
