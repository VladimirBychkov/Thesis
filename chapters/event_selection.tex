%%%%%%%%%%%%%%%%%%%%%%%%%%%%%%%%%%%%%%%%%%%%%%%%%%%%%%%%%%%%%%%%%%%%%%%%%%%%%%%%
% event_selection.tex: Select of showering and tracking events:
%%%%%%%%%%%%%%%%%%%%%%%%%%%%%%%%%%%%%%%%%%%%%%%%%%%%%%%%%%%%%%%%%%%%%%%%%%%%%%%%
\chapter{Contained and Escaped $\nu_\mu$ CC Event Selection}
\label{event_selection_chapter}
%%%%%%%%%%%%%%%%%%%%%%%%%%%%%%%%%%%%%%%%%%%%%%%%%%%%%%%%%%%%%%%%%%%%%%%%%%%%%%%%

In this chapter, algorithms for event selection are
considered. The first part of the chapter is dedicated to the contained sample and its selection 
criteria. These criteria were developed and optimazed by people from NuMu group in NOvA collaboration 
which resulted in several Ph.D. theses. For completeness the criteria are briefly summarized in
section \ref{cont_sec}.

The second part of the chapter \ref{esc_sec} presents development of selection criteria for
the escaping sample - events which are failed to get selected into contained sample and 
partially left the detector. The section is supported with numerious plots which show how 
different selections help to reduce initially gigantic cosmic background down to small fraction
of total selected events. The final selections is chosen to get maximal improvement in 
$\theta_{23}$ and $\Delta m_{32}^2$ sensitivity and to remove any bias optimization is performed
on simulated events at the far detector. To study cosmic background a cosmic trigger data is used.

The escaping sample on its own does not provide a good measurements of $\theta_{23}$ and 
$\Delta m_{32}^2$, however it improves measurements when combined with contained sample. And this
is the main goal of this thesis. 

\section{Contained Sample} \label{cont_sec}
Contained sample is selected with the help of series cuts which ensure that passed events
are $\nu_\mu$ CC events with good energy resolution and low cosmic contamination. Several
next subsecitions are briefly describes these cuts.

\subsection{Data Quality}
Data quality is the first cut in contained sample selection flow and serves to remove
events of poor quality. Events which pass data quality cut have
\begin{itemize}
\item more than 20 reconstructed hits. There is no useful inforamtion for events with fewer 
number of hits.
\item at least one track reconstructed by cosmic tracker. This criterion confirms that there is
cluster of hits which could be associated with track.
\item PID value of ReMId selector should be greater than zero. This ensures that at least on 
3D Kalman track was reconstructed.
\item slice which spans more than 4 cosecutive planes. This cut removes events with geometric 
gaps and highly vertical tracks, thus the cut is a good preselector for a good fraction of 
cosmic rays.
\end{itemize}

\subsection{Containment} \label{Containment}
The main purpose of containment cut is to select events with activity happening inside the 
detector. This allows partially to remove cosmic rays since they enter detector from the top 
or the side edges. Contained events have the best energy resolution and satisfy following 
criteria
\begin{itemize}
\item Kalman Forward Cell $>$ 6,
\item Kalman Backward Cell $>$ 6,
\item Cosmic Forward Cell $>$ 0,
\item Cosmic Backward Cell $>$ 7,
\item Planes to Front $>$ 1,
\item Planes to Back $>$ 1,
\item Minimal Prong Distance to FD Top $>$ 60,
\item Minimal Prong Distance to FD Bottom $>$ 12,
\item Minimal Prong Distance to FD East $>$ 16,
\item Minimal Prong Distance to FD West $>$ 12,
\item Minimal Prong Distance to FD Front $>$ 18,
\item Minimal Prong Distance to FD Back $>$ 18.
\end{itemize}
First four criteria use tracks reconstructed with the help of Kalman Tracker and Cosmic Tracker.
For every track, number of projected cells\footnote{along the track} from start/end of the track 
to the edge of the detector is estimated and these numbers should be bigger than above stated
tresholds. The next two criteria checks that reconstructed slice does not have hits in the first
and the last planes of detector active region\footnote{Data were taken during detector construction
period, thus active region was changing with time.}. Finally, the last six criteria use objects
reconstructed by Fuzzy clustering algorithm, the objects consist of hits that could belong to 
different clusters (in other words, two particle which move in close trajectories could contribute
to one hit). The tresholds in the last six criteria are result of optimization carried out in 
\cite{numucont_technote}.

\subsection{$\nu_\mu$ CC selection} \label{CCSel}
This cut aims to select $\nu_\mu$ CC like events and to reject $\nu_e$ CC and $\nu$ NC like events.
Extensive optimization analysis is done in \cite{numupid_technote} and result are folowong
\begin{itemize}
\item ReMId $>$ 0.5 - event has long muon track with a good ReMId score,
\item CVN $>$ 0.5 - event has a good score for $\nu_\mu$ CC classifier based on convolutional neural net.
\end{itemize}

\subsection{Cosmic Rejection}
The final cut is cosmic rejection and its purpose is to remove cosmic events which might slipped in after
previous cuts. Several variables are extracted from the event and input into Boosted Decision Tree (BDT)
algorithm\footnote{See next chapter for more information about BDT algorithm.}\cite{cosrej_technote}. 
These features are
\begin{itemize}
\item cosine of angle between muon track and incoming neutrino,
\item reconstructed y-deriction of the muon track,
\item reconstructed length of the muon track,
\item largest y-position of either the start or the end of muon the track,
\item CVN cosmic score,
\item min(CosmicForwardCell + CosmicBackwardCell, KalmanForwardCell + KalmanBackwardCell), this 
variable helps to descriminate going-through cosmic muons which passed all previous cuts,
\item ratio of hit number in kalman track to hit number in slice, this ratio is a measure of hadronic
energy fraction. Cosmic events have very little hadronic energy as compared to beam neutrino events. 
\end{itemize}
BDT algorithm outputs a PID value and the cut choice is based on optimization carried out in 
\cite{numupid_technote}. Events pass cosmic rejection selection if PID value is grater than $0.5$.

\subsection{Resolution Bins}
Previous analysis carried out in \cite{Kanika}, \cite{Dominick}, \cite{Susan} used one selected sample for 
fitting to actual data. Analysis carried out in \cite{Nick} used several samples for fitting to data 
such as contained QE sample, contained non-QE sample and escaping sample\footnote{Most of the events 
selected in escaping sample in \cite{Nick} have energy far from first oscillation dip, thus it is not 
contributing much to the disappearance analysis. Current analysis aims to address this issue.}. This 
approach separates events into different energy resolution bins and there by improves final sensitivity 
of the experiment. Finally, recent analysis \cite{Luke} goes further and separates contained sample into 
four independent ones with different fraction of hadronic energy in selected events. 

Hadronic energy fraction ($\frac{E_{had}}{E_\nu}$) distribution in made for every reconstructed energy
bin and than it splitted into four equal parts - lowest, second to lowest, second to highest and highest
quantiles - so every quantile has equal number of events. Plot of $\frac{E_{had}}{E_\nu}$ vs $E_\nu$ and 
quantile bondaries are shown in \ref{fig:hadE_Quant}. Since detector configuration was different for 
different run periods, the analysis uses different quantiles boundaries to account for that. 
\begin{figure}[h]
\centering
\includegraphics[width=0.9\textwidth]{figures/hadERecoE_QuantileLimits.png}
\caption{Haronic energy fraction quantiles.}
{Hadronic energy fraction vs reconstructed neutrino energy plot allows to determine for every bin of
reconstructed energy hadronic energy fraction quantile. Events from every bin are splitted into 4 parts
and separated by black horizontal lines. This plot is made for run period 2, plots for otherrun periods 
are similar.}
\label{fig:hadE_Quant}
\end{figure}

\section{Escaping Sample} \label{esc_sec}
The rest of the chapter is an original work devoted to selection of escaping sample with minimal
cosmic backrgound and maximal beam neutrino signal.

\subsection{Definition and Basic Selection} \label{base_cut}
Escaping sample does not have a nice quality of contained sample such as low cosmic background and
high energy resolution nevertheless it might provide additional information on neutrino oscillation
effectively increasing gathered statistics. 

Events which define escaping sample do not need to be consist of particles which leave the detector but
rather ones which fail to pass some cuts of contained sample. The obvious cut escaping events should 
fail is containment cut \ref{Containment}. Escaping events also should pass some cuts from contained 
sample because it is still needed that they contain useful information for the analysis. Those cuts are
\begin{itemize}
\item data quality cut - events have reasonable number of reconstructed hits and tracks,
\item CVN $>$ 0.5 - events are highly $\nu_\mu$ CC events,
\item loose cosmic rejection, cosmic rejection PID value is $>$ 0.48 - events should not to be an obvious 
cosmic rays.
\end{itemize}

Selections above are the basic criteria for the escaping sample. Figure \ref{fig:length_1} illustrates 
expected number of cosmic and signal events which pass basic criteria for primary track length distribution.
\begin{figure}[h]
\centering
\includegraphics[width=0.9\textwidth]{figures/reco_Length_Base.pdf}
\caption{Length of Primary Track After Base Cuts.}
{The plot illustrates almost two order of magnitude difference between expected number of signal events 
(black) and cosmic events (red). }
\label{fig:length_1}
\end{figure}

\subsection{Preselection} \label{presel}
The next step is to understand what variables could help to descriminate cosmic rays and beam neutrino
signal. The first approach here is to try to determine direction of the primary track - is track going 
out of or in the detector? Unfortunately, the time resolution of individual hits is not enough to determine 
confidently the direction of the track and other approaches are needed. 
\begin{figure}[h]
\begin{subfigure}[t]{0.9\textwidth}
  \centering
  \includegraphics[width=1.0\linewidth]{figures/numu_and_cosmic_top.png}
  \caption{Topologies.}
  {Dash lines illustrates parts which are not visible in the detector.}
  \label{fig:cosmic_top}
\end{subfigure}
\vspace{0.5cm}
\newline
\begin{subfigure}[t]{0.9\textwidth}
  \centering
  \includegraphics[width=1.0\linewidth]{figures/numu_and_cosmic_top_2.png}
  \caption{Reconstracted Kalman tracks.}
  \label{fig:cosmic_top_2}
\end{subfigure}
\caption{Cosmic rays and beam neutrinos in the far detector.}
\label{fig:top_and_reco_tracks}
\end{figure}

\begin{wrapfigure}{r}{0.5\textwidth}
\vspace{-15pt}
  \begin{center}
    \includegraphics[width=0.5\textwidth]{figures/backdist.png}
  \end{center}
\vspace{0.5mm}
\caption{Backwards projected distance}
\label{fig:bakdist}
\end{wrapfigure}
Figure \ref{fig:cosmic_top} shows topologies of cosmic rays and beam neutrinos with escaping muons and figure
\ref{fig:cosmic_top_2} illustrates corresponding reconstructed Kalman tracks. The feature of the Kalman 
tracker is that it all reconstructed tracks are asumed to point along the positive Z-axis, this allows 
to remove a big fraction of cosmic ray at a small expense to the signal in the following way. All cosmic 
rays effectively are devided into two populations, the one where track direction assumed correctly and the 
one where it is not. These populations are clearly seen on the figure \ref{fig:stopdiry_1} where distribution
of reconstructed Kalman track Y-direction at its end (StopDirY) is showed for the cosmic rays and the primary 
track of the signal events. The usage of this variable will become clear later on. Now, one can see that it 
is easy to remove one population by requiring backwards projected distance \ref{fig:bakdist} (KalBakCell) to 
be greater than some treshold. The actual cut is perform for 'number of backwards projected cells' variable, 
see figure \ref{fig:kalbakcell}.
\begin{figure}[h]
\centering
\includegraphics[width=0.9\textwidth]{figures/reco_StopDirY_Base.pdf}
\caption{Primary Track's End Y-Direction.}
{Two population structure is clearly visible for cosmic rays background, this is an artifact of Kalman tracker.
Figure \ref{fig:cosmic_top_2} shows membres of these populations. The tracker reconstructs a track with an assumption 
that its starting point is always to the right of its end point. }
\label{fig:stopdiry_1}
\end{figure}
\begin{figure}[h]
\centering
\includegraphics[width=0.9\textwidth]{figures/reco_KalbakCell_Base.pdf}
\caption{Number of Backwards Projected Cells.}
{Three order of magninture more cosmic events for lower end of number of backwards projected cells comes from cosmic 
rays which move along the positive Z-axis. See left example of cosmic ray in figure \ref{fig:cosmic_top_2}. }
\label{fig:kalbakcell}
\end{figure}

The next cut does not designed to further decrease amount of cosmic rays but rather make selected sample 
to be more clean for energy estimation. Next chapter is completely devoted to this topic. It is required for 
hadronic cluster (all hits in slice except the ones which belong to primary track) that a hit which is the 
closest hit to the edge of detector was no closer to the edge than 2 cells. Figure \ref{fig:stopdiry_2} shows
distribution of StopDirY after this cut and the previous one (KalBakCell $>$ 7).
\begin{figure}[h]
\centering
\includegraphics[width=0.9\textwidth]{figures/reco_StopDirY_Base_kbc7_ncfe2.pdf}
\caption{Primary Track's End Y-Direction after two preselection cuts.}
{Cut on number of cells from the edge to hadronic cluster has a small impact on all distributions, it is needed
only for a clean sample to train energy estimator on. On the opposite, Kalbakcell cut removes half of the cosmic
background. }
\label{fig:stopdiry_2}
\end{figure} 

There is another variable which remove big fraction of cosmic rays similar to KalBakCell. And this variable 
is StopDirY. As can be seen from figure \ref{fig:cosmic_top_2}, all cosmic rays, which have incorrectly reconstructed
direction, have their StopDirY grater than zero. By asking that variable to be less than zero for escaping 
sample the second population of cosmic rays could be eliminated. Physically, that means that only those events
are selected whose primary track stops or leaves the detector pointing downward. This cut has a big impact
on the signal as incoming beam neutrinos pointing upward at a few degrees level, thus muons resulting in $\nu_\mu$
CC interaction also prefer upward direction. Track direction at its end is more useful variable as compared to
track's start since it has information about cosmic ray direction as ray enters the detector. 

Figure \ref{fig:length_2} shows that escaping sample still has noticeable cosmic ray background and the next 
section developes a few more optimized cuts to decrease further the rest of background.

\begin{figure}[!h]
\centering
\includegraphics[width=0.9\textwidth]{figures/reco_Length_Presel.pdf}
\caption{Length of Primary Track After Base and Preselection Cuts.}
{The plot illustrates abundance of cosmic ray background after base and preselection cuts. }
\label{fig:length_2}
\end{figure}

\subsection{Selection}
Following the section \ref{CCSel} for contained sample, $\nu_\mu$ CC selection for escaping sample is exactly 
the same, namely ReMid and CVN score have to be grater than 0.5. Figure \ref{fig:remid} shows ReMId value for
events which pass Base \ref{base_cut} and Preselection \ref{presel} cuts. 
\begin{figure}[h]
\centering
\includegraphics[width=0.9\textwidth]{figures/reco_remid_kbc7_ncfe2_diry0.pdf}
\caption{ReMId value for the events which pass Base and Preselection cuts.}
{Cut on ReMId value is chosen similar to contained sample. Low value means a poorly reconstructed muon track or
track produced by a pion. }
\label{fig:remid}
\end{figure}

The last two criteria are based on the facts that cosmic rays tend to be vertical and the majority of them are 
muons with small amount of deposited energy outside the reconstructed track. Figures \ref{fig:ptp} and \ref{fig:calHadE}
illustrate previous statement - the first figure shows reconstructed transverse momentum over total momentum 
($\frac{p_t}{p}$) and the second one shows reconstructed hadronic energy of the events after Base and Preselection 
cuts. However, a simple cut on event hadronic energy leads to a significant loss in signal and as result to a 
worse experiment sensitivity for $\theta_{23}$ and $\Delta m^2_{32}$ parameters. To bypass this unpleasantness one 
can use a strength of CVN classifier - it predicts not only type of neutrino but also a type of interaction the 
neutrino participated in. Figure \ref{fig:cvnVsHadE} shows distribution of reconstructed hadronic energy for 
different type of neutrino interaction determined by CVN classifier. The final cut optimization is done for 
$\frac{p_t}{p}$ variable and reconstructed hadronic energy of CVN QE-like events.

The ultimate goal of this work is to measure $\theta_{23}$ and $\Delta m^2_{32}$ parameters which govern muon
neutrino oscillations and the tuning of selection criteria is done to maximaze experiment sensitivity. However,
to measure a sensitivity one needs to estimate energy for the selected events and to make energy estimator one 
needs final selection criteria. To break this vicious circle energy estimator (see chapter \ref{energy_estimator_chapter})
is trained on the events which pass not final but rather close to final selection criteria. And then, final 
selection are chosen by using that energy estimator.
\begin{figure}[h]
\centering
\includegraphics[width=0.9\textwidth]{figures/reco_ptp_Presel_remid5.pdf}
\caption{Reconstructed $\frac{p_t}{p}$.}
{Distribution of reconstructed transverse momentum over total momentum for events which pass Base, Preselection and
ReMId grater than 0.5 cuts. }
\label{fig:ptp}
\end{figure}
\begin{figure}[h]
\centering
\includegraphics[width=0.9\textwidth]{figures/reco_hadCalE_Presel_remid5.pdf}
\caption{Hadronic Calorimetric Energy.}
{Distribution of hadronic calorimetric energy for events which pass Base, Preselection and ReMId grater than 0,5 cuts. 
Simple cut on this variable sagnificantly hurts sensitivity, thus different approach is taken.}
\label{fig:calHadE}
\end{figure}
\begin{figure}[!th]
\begin{subfigure}[t]{0.95\textwidth}
  \centering
  \includegraphics[width=1.0\linewidth]{figures/reco_CVNInt-vs-hadCalE_Presel_remid5.pdf}
  \caption{Signal.}
  \label{fig:cvnVsHadE_sig}
\end{subfigure}
\vspace{0.5cm}
\newline
\begin{subfigure}[t]{0.95\textwidth}
  \centering
  \includegraphics[width=1.0\linewidth]{figures/cosmic_reco_CVNInt-vs-hadCalE_Presel_remid5.pdf}
  \caption{Cosmic Background.}
  \label{fig:cvnVsHadE_bkg}
\end{subfigure}
\caption{ Hadronic Calorimetric Energy for Different CVN Classes.  }
\label{fig:cvnVsHadE}
\end{figure}

\begin{figure}[h]
\centering
\includegraphics[width=0.9\textwidth]{figures/reco_hadCalE_Presel_CVNQE.pdf}
\caption{Hadronic Calorimetric Energy for CVN QE-like events.}
{Distribution of hadronic calorimetric energy for events which pass Base, Preselection, ReMId grater than 0,5 cuts and 
classified by CVN as QE events. Optimized cut was chosen at 0.1 GeV. }
\label{fig:calHadEVNQE}
\end{figure}

The final selections after optimization are
\begin{itemize}
\item $\frac{p_t}{p} < 0.6$,
\item hadronic energy for CVN QE-lie events is less than 0.1 GeV.
\end{itemize}
Figures \ref{fig:startx}, \ref{fig:starty}, \ref{fig:startz} show X, Y, and Z-position of primary track start 
(position of interaction point), figures \ref{fig:startXZ}, \ref{fig:startYZ}, \ref{fig:startYX}  show distributions 
of primary track start in planes X-Z, Y-Z, X-Y, figure \ref{fig:length_3} shows length of the primary track for 
signal and cosmic events. All these figures show events whic pass full escaping sample selection criteria. 

\begin{figure}[!th]
\centering
\includegraphics[width=1.0\textwidth]{figures/reco_StartX_Full.pdf}
\caption{StartX Position of Primary Track.}
{Distribution of primary track Start X-position for events which pass all selection criteria.  }
\label{fig:startx}
\end{figure}

\begin{figure}[!th]
\centering
\includegraphics[width=1.0\textwidth]{figures/reco_StartY_Full.pdf}
\caption{StartY Position of Primary Track.}
{Distribution of primary track Start Y-position for events which pass all selection criteria.  }
\label{fig:starty}
\end{figure}

\begin{figure}[!th]
\centering
\includegraphics[width=1.0\textwidth]{figures/reco_StartZ_Full.pdf}
\caption{StartZ Position of Primary Track.}
{Distribution of primary track Start Z-position for events which pass all selection criteria.  }
\label{fig:startz}
\end{figure}

\begin{figure}[!th]
\centering
\begin{subfigure}[t]{0.95\textwidth}
  \centering
  \includegraphics[width=1.0\linewidth]{figures/reco_trkStartX-vs-trkStartZ_Full.pdf}
  \caption{Signal.}
  \label{fig:startXZ_sig}
\end{subfigure}
\vspace{0.5cm}
\newline
\begin{subfigure}[t]{0.95\textwidth}
  \centering
  \includegraphics[width=1.0\linewidth]{figures/cosmic-reco_trkStartX-vs-trkStartZ_Full.pdf}
  \caption{Cosmic Background.}
  \label{fig:startXZ_bkg}
\end{subfigure}
\caption{ Distribution of Primary Track Start in X-view.}
{ Events pass all the escaping sample selection criteria. X-view can be interpreted as look from the top of the detector. }
\label{fig:startXZ}
\end{figure}

\begin{figure}[!th]
\centering
\begin{subfigure}[t]{0.95\textwidth}
  \centering
  \includegraphics[width=1.0\linewidth]{figures/reco_trkStartY-vs-trkStartZ_Full.pdf}
  \caption{Signal.}
  \label{fig:startYZ_sig}
\end{subfigure}
\vspace{0.5cm}
\newline
\begin{subfigure}[t]{0.95\textwidth}
  \centering
  \includegraphics[width=1.0\linewidth]{figures/cosmic-reco_trkStartY-vs-trkStartZ_Full.pdf}
  \caption{Cosmic Background.}
  \label{fig:startYZ_bkg}
\end{subfigure}
\caption{ Distribution of Primary Track Start in Y-view.}
{ Events pass all the escaping sample selection criteria. Y-view can be interpreted as look from the side of the detector. }
\label{fig:startYZ}
\end{figure}

\begin{figure}[!th]
\centering
\begin{subfigure}[t]{0.95\textwidth}
  \centering
  \includegraphics[width=1.0\linewidth]{figures/reco_trkStartY-vs-trkStartX_Full.pdf}
  \caption{Signal.}
  \label{fig:startYX_sig}
\end{subfigure}
\vspace{0.5cm}
\newline
\begin{subfigure}[t]{0.95\textwidth}
  \centering
  \includegraphics[width=1.0\linewidth]{figures/cosmic-reco_trkStartY-vs-trkStartX_Full.pdf}
  \caption{Cosmic Background.}
  \label{fig:startYX_bkg}
\end{subfigure}
\caption{ Distribution of Primary Track Start in X-Y view.}
{ Events pass all the escaping sample selection criteria. X-Y view can be interpreted as look from the side of the detector
along the beam direction. }
\label{fig:startYX}
\end{figure}

\begin{figure}[!th]
\centering
\includegraphics[width=1.0\textwidth]{figures/reco_Length_Full.pdf}
\caption{Length of Primary Track.}
{ Events pass all the escaping sample selection criteria. }
\label{fig:length_3}
\end{figure}

\clearpage
\subsection{Summary}
In this section all selection criteria are briefly stated together. 

\textbf{Base cuts}
\begin{itemize}
\item data quality,
\item failed containment cut for the contained sample,
\item CVN $>$ 0.5,
\item cosmic rejection PID value is $>$ 0.48.
\end{itemize}

\textbf{Preselection}
\begin{itemize}
\item number of backwards projected cells for primary track is grater than 7,
\item number of cells from detector edge to hadronic cluster is grater than 2,
\item Y-component of track's direction at its end is less than 0.
\end{itemize}

\textbf{Selection}
\begin{itemize}
\item ReMId is grater than 0.5,
\item $\frac{p_t}{p}$ is less than 0.6,
\item hadronic energy for CVN QE-like events is grater than 0.1 GeV.
\end{itemize}

This concludes escaping and contained events selection chapter, the next step is to determine selected events energies.
%%%%%%%%%%%%%%%%%%%%%%%%%%%%%%%%%%%%%%%%%%%%%%%%%%%%%%%%%%%%%%%%%%%%%%%%%%%%%%%%
