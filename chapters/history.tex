%%%%%%%%%%%%%%%%%%%%%%%%%%%%%%%%%%%%%%%%%%%%%%%%%%%%%%%%%%%%%%%%%%%%%%%%%%%%%%%%
%history.tex: Chapter on neutrino history and neutrino oscillation discovery:
%%%%%%%%%%%%%%%%%%%%%%%%%%%%%%%%%%%%%%%%%%%%%%%%%%%%%%%%%%%%%%%%%%%%%%%%%%%%%%%%
\chapter{Neutrino and its oscillation}
\label{history_chapter}
%%%%%%%%%%%%%%%%%%%%%%%%%%%%%%%%%%%%%%%%%%%%%%%%%%%%%%%%%%%%%%%%%%%%%%%%%%%%%%%%

\section{History of neutrino discoveries}
The energy spectrum of electrons in beta decay was first accurately measured by James Chadwick 
in 1914 \cite{chadwick}.  Previous attempts by Lise Meitner and Otto Hahn in 1911 and 
Jean Danysz in 1913 showed hints that the electron energy spectrum was continuous. 
At the time this was an obvious contradiction with laws of energy and angular momentum 
conservation. A nucleus emitting an electron changes its state and energy of the 
electron should be equal to the energy difference of two nucleus' states. After 
several attempts to explain the mystery of beta decay Wolfgang Pauli postulated 
\cite{pauli} that beta decay is actually a three particle decay. He called the third 
particle ``neutron" - an electrically neutral and very light particle - and it was 
responsible for carrying away some portion of the energy which leads to continuous 
electron energy spectrum. Two years later in 1932 an actual neutron - the proton's 
partner inside a nucleus - was discovered by Chadwick \cite{chadwick2}.  To avoid 
confusion Enrico Fermi proposed a new name for Pauli's particle \cite{fermi} - 
neutrino, which means ``little neutral one".

It took more than 20 years for the first neutrino to be observed.  In 1956 Reines 
and Cowan made a direct observation of a neutrino \cite{cowan}, whose flavor was 
later recognized as an electron anti-neutrino $\bar{\nu}_e$. An anti-neutrino was 
produced in a nuclear fission reactor via neutron decay
\be
n \rightarrow p + e^- + \bar{\nu}_e.
\ee
An anti-neutrino interacted with a proton in a detector and produced positron and neutron
\be
\bar{\nu}_e + p \rightarrow e^+ + n. \lb{nue1}
\ee
The signature of both processes, and thus evidence of the neutrino, was observed using 
a detector in the following way. An anti-neutrino interacted with a proton to produce 
a positron and neutron.  The positron annihilated with an electron and produced two 
photons.  Shortly after the neutron was captured by a nucleus and emitted several photons.  
Two photons followed by a third one gave an indication of a neutrino interaction inside 
the detector. The Nobel Prize was awarded to Reines in 1995 for the detection of the neutrino.

In 1962 the neutrino family was expanded by observation of the second type of neutrino - 
a muon neutrino $\nu_\mu$.  Leon Lederman, Melvin Schwartz, and Jack Steinberger used 
the world's largest accelerator at that time at Brookhaven National Laboratory to direct protons onto 
a fixed target to produce charged pions. Those pions subsequently decayed to 
(anti)muons and muon [anti]neutrinos. At the end of pion decay pipe plates made of steel 
and lead were installed to absorb muons. Beyond the metal absorbers a spark chamber was 
placed to identify neutrino interactions similar to \p{nue1},
\be
\nu_\mu + n \rightarrow \mu^- + p,\qquad
\bar{\nu}_\mu + p \rightarrow \mu^+ + n.
\ee
After processing all experimental data, 34 neutrino interactions were found with 
a single muon track in the spark chamber \cite{danby}. Absence of observed electrons in 
the chamber proved that $\nu_\mu \ne \nu_e$ and lead the fact that in 1988 the group was awarded 
the Nobel Prize for their discovery of the muon neutrino.

The discovery of weak gauge bosons at CERN by UA1 and UA2 collaborations \cite{arnison}, 
\cite{arnison2} in 1983 provided new ways to study particles interacting via weak bosons. 
Careful measurements of the $Z$ boson decay rates helped to establish the total number 
of neutrinos which interact through $W^\pm$ and $Z$ bosons. ALEPH detector on the 
CERN Large Electron Positron collider determined \cite{decamp} in 1989 that only 
three different neutrinos participate in weak interactions. Sterile neutrinos, which 
do not interact weakly, have not been found but are still being searched for.

The ALEPH experiment suggested that a third neutrino should be associated with the tau 
lepton, which was discovered in 1975 \cite{perl}. The Direct Observation of Nu Tau (DONUT) 
experiment was setup at Fermilab in 2000 and looked for reactions
\be
\nu_\tau + n \rightarrow \tau^- + p,\qquad
\bar{\nu}_\tau + p \rightarrow \tau^+ + n.
\ee
The general idea of the DONUT experiment was similar to the experiment conducted at 
Brookhaven National Lab in the early 1960s. A proton beam was directed into a fixed 
target resulting in a shower of meson and baryons. In order to decrease the background 
of $\nu_\mu$ neutrinos, deflecting magnets were installed along the decay pipe to remove 
muons which produce $\nu_\mu$ neutrinos upon decaying. Tau neutrinos were produced through 
$D_S \rightarrow \tau + \nu_\tau$ process, and as many as 4 tau neutrino interactions 
were observed \cite{kodama}. This number of interactions was sufficient to confirm 
the existence of the tau neutrino.

\section{Neutrino oscillations}
The idea of neutrino oscillations was suggested in the work of Bruno Pontecorvo in 
1957 \cite{pontecorvo}. Although at the time only one neutrino flavor was known, 
Bruno described the mixing of electron neutrinos and antineutrinos. Later, when the muon 
neutrino was discovered, the physicists Maki, Nakagawa, and Sakata used Pontecorvo's 
framework \cite{maki} to describe neutrino flavor oscillations. After the discovery 
of the tau neutrino, a third neutrino was added to the oscillation picture.

Before Ray Davis and collaborators' observation of solar neutrino deficit in the 1960's 
\cite{davis}, neutrino oscillations were considered possible but hardly realistic. 
Using the standard solar model the total number of electron neutrino coming from the sun 
was predicted by John Bahcall \cite{bahcall}. However, an obvious deficit of solar 
neutrinos was measured. Later, more precise measurements of solar neutrino flux were 
made in Japan - Super-Kamiokande \cite{suzuki}, and in Sudbury Neutrino Observatory 
\cite{mcdonald} in Canada. Both observations matched prior results, but contradicted predictions. 
This solar neutrino problem is elegantly explained in the framework of neutrino oscillations. 
In the following section a theoretical description of neutrino oscillations will be given.
%%%%%%%%%%%%%%%%%%%%%%%%%%%%%%%%%%%%%%%%%%%%%%%%%%%%%%%%%%%%%%%%%%%%%%%%%%%%%%%%

%%%%%%%%%%%%%%%%%%%%%%%%%%%%%%%%%%%%%%%%%%%%%%%%%%%%%%%%%%%%%%%%%%%%%%%%%%%%%%%%
