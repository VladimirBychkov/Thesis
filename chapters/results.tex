%%%%%%%%%%%%%%%%%%%%%%%%%%%%%%%%%%%%%%%%%%%%%%%%%%%%%%%%%%%%%%%%%%%%%%%%%%%%%%%%
% results.tex:
%%%%%%%%%%%%%%%%%%%%%%%%%%%%%%%%%%%%%%%%%%%%%%%%%%%%%%%%%%%%%%%%%%%%%%%%%%%%%%%%
\chapter{Results}
\label{results_chapter}
%%%%%%%%%%%%%%%%%%%%%%%%%%%%%%%%%%%%%%%%%%%%%%%%%%%%%%%%%%%%%%%%%%%%%%%%%%%%%%%%

After analysis software processed the Far Detector data, 126 events were selected for 4 contained 
samples, including an expected background of 5.8 misidentified cosmic rays and 3.4 misidentified 
neutrino events of other types. For the escaping sample, XXX events were selected, including an expected
background of 6.0 misidentified cosmic rays and 0.4 misidentified neutrino of other types. The resulting
5 spectra were fit simultaneously with the best-fit parameters are $\sin^2\theta_{23} = 0.567$ and 
$|\Delta m^2_{32}| = 2.45\times 10^{-3} eV^2$.

Predicted 21, sel - 32

\section{Far Detector Selected Samples}

\section{Event Displays of Several Selected Escaping Events}
Figures \ref{fig:candidate1}, \ref{fig:candidate2} and \ref{fig:candidate3} give the examples of selected escaping 
events at the Far Detector.
\clearpage
\begin{figure}[!th]
\centering
\includegraphics[width=1.3\textwidth, angle=90, origin=c]{figures/evd_xzyz-proj_16787_201283.pdf}
\caption{Selected FD escaping $\nu_\mu$ CC candidate}
{(Rotated left) The X vs. Z and Y vs. Z views of selected FD escaping $\nu_\mu$ CC candidate with estimated muon 
neutrino energy of 2.54 GeV.}
\label{fig:candidate1}
\end{figure}

\clearpage
\begin{figure}[!th]
\centering
\includegraphics[width=1.3\textwidth, angle=90, origin=c]{figures/evd_xzyz-proj_19279_533637.pdf}
\caption{Selected FD escaping $\nu_\mu$ CC candidate}
{(Rotated left) The X vs. Z and Y vs. Z views of selected FD escaping $\nu_\mu$ CC candidate with estimated muon
neutrino energy of 2.78 GeV.}
\label{fig:candidate2}
\end{figure}

\clearpage
\begin{figure}[!th]
\centering
\includegraphics[width=1.3\textwidth, angle=90, origin=c]{figures/evd_xzyz-proj_19317_706081.pdf}
\caption{Selected FD escaping $\nu_\mu$ CC candidate}
{(Rotated left) The X vs. Z and Y vs. Z views of selected FD escaping $\nu_\mu$ CC candidate with estimated muon
neutrino energy of 1.85 GeV.}
\label{fig:candidate3}
\end{figure}

\clearpage
\section{Results of the Fit}
The Far Detector data selected for all 5 spectra were fit to the extrapolated prediction. The fit included penalty to 
account for the systematic uncertainties according to equation \ref{chi2syst}. The figures \ref{fig:dataprediction_cont}
and \ref{fig:dataprediction_uncont} show overlaid Far Detector data with the predicted $1\sigma$ systematic uncertainty band
for the four contained and one escaping spectra respectively.  

\begin{figure}[!th]
\centering
\includegraphics[width=1.0\textwidth]{figures/fd_dataprediction_spectra__fhc_quant14__2018calc.pdf}
\caption{Observed 4 contained spectra at the Far Detector with best fit and systematics uncertainties}
{}
\label{fig:dataprediction_cont}
\end{figure}

\begin{figure}[!th]
\centering
\includegraphics[width=1.0\textwidth]{figures/fd_prediction_data_escaping_spectra__fhc__2017calc.pdf}
\caption{Observed escaping spectrum at teh Far Detector with best fit and systematics uncertainties}
{}
\label{fig:dataprediction_uncont}
\end{figure}

\begin{figure}[!th]
\centering
\includegraphics[width=1.0\textwidth]{figures/contours_realdata_cosmics__extrap_systs_2018calc_nh_syststat__numu2018.pdf}
\caption{90\% confidence interval with and without systematic uncertainties}
{Confedence interval with systematic uncertainties is shown in blue and without systematics uncertainties is shown in red. 
The best-fit parameters are $\sin^2\theta_{23} = 0.567$ and $|\Delta m^2_{32}| = 2.45\times 10^{-3} eV^2$.}
\label{fig:cant_w_wo_syst}
\end{figure}

%%%%%%%%%%%%%%%%%%%%%%%%%%%%%%%%%%%%%%%%%%%%%%%%%%%%%%%%%%%%%%%%%%%%%%%%%%%%%%%%
