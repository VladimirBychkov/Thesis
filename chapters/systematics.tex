%%%%%%%%%%%%%%%%%%%%%%%%%%%%%%%%%%%%%%%%%%%%%%%%%%%%%%%%%%%%%%%%%%%%%%%%%%%%%%%%
% systematics.tex:
%%%%%%%%%%%%%%%%%%%%%%%%%%%%%%%%%%%%%%%%%%%%%%%%%%%%%%%%%%%%%%%%%%%%%%%%%%%%%%%%
\chapter{Systematics}
\label{systematic_chapter}
%%%%%%%%%%%%%%%%%%%%%%%%%%%%%%%%%%%%%%%%%%%%%%%%%%%%%%%%%%%%%%%%%%%%%%%%%%%%%%%%

Precise measurements of the oscillation parameters depend on correct extrapolated predictions of neutrino spectrum at 
the Far Detector. In turn, extrapolated predictions depend on our understanding of all properties of the detectors as
well as properties of particles propagation and interaction with other particles. Naturally this understanding cannot 
be absolute and the lack of it leads to uncertainties in the final result, this uncertainties are called systematic 
unsertainty, as opposed to statistical uncertainties which comes from probabalistic nature of high energy experiment.
While statistical uncertainties are mitigated by longer exposure, systematic uncertainties could be mitigated by the
experiment design. For example, two detectors design of NOvA reduces an impact of flux and cross section uncertainties.

The first section of this chapter describes the treatment of systemtic uncertainties in NOvA and the following ones
discuss the impact of different systematic uncertainties on extrapolated predictions. 

\section{Evaluation of Systematic Uncertainties}
For some systematics analysis framework allows a reweightable approach to evaluate an impact of the specific systematic
on the analysis result. This approach helps to reduce computational resources \cite{cafana}. For instance, any cross section 
parameter has it own uncertainty and for every simulated event a weight is calculated based on that uncertainty and 
truth information or the output of reconstructed algorithms, thus one gets a new (shifted) prediction which 
corresponds to that cross section parameter. One gets 2 shifted predictions by using weights which are calculated for 
$\pm 1\sigma$ shifts for some cross section parameter; nominal and 2 shifted predictions are used to produced a 
systematic error band on all plots which are presented in this chapter. Moreover, for the sake of space systematic 
error bands are shown only for escaping sample, study of systematic uncertainties for contained sample can be found in
\cite{Luke}.

Those systematic effects which need to alter information on hit-by-hit basis cannot be used in reweightable scheme 
and a completely new set of Monte Carlo simulation files is generated, for instance, new sets of files are requered for 
systematically shifted scintillation model, calibration systematics etc. 

During the fitting procedure \ref{chi2syst} fitter needs to know prediction for fractional systetamic shift. This is 
achieved by using a cubic spline interpolation between nominal, $\pm 1\sigma$ and $\pm 2\sigma$ shifts. 

Analysis framework distincts two types of systematic uncertainties in NOvA, those which are the same or correlated for 
both detectors and those which are independent. Correlated systematic uncertainties shift prediction in both detectors
simultaniously by the same amount and are called \textit{absolute} uncertainties. Flux and neutrino cross section 
uncertainties are examples of absolute uncertainties. Systematic effects which are independent across detectors 
shift prediction by the same amount but in opposite direction and are called \textit{relative} uncertainties.  

\section{Cross Section Unsertainty}
GENIE cross section models depends on 80+ parameters and uncertaintes on all these paramters allow to calculate cross
section weights \cite{GENIE}. When one alters some parameter weight is determined as ratio of cross section with altered 
parameter to nominal one. Later, at analysis time, every selected event at the Near or the Far Detector is weighted with 
above calculated weight, thus shifted prediction is generated. The marginalization process\footnote{fitting procedure with 
systematic uncrtainties taken into account.} is computationally heavy and only those GENIE cross section systematics are 
used which has significant impact on the analysis. Table \ref{table:XSecParam} summarize cross section parameters used in
the analysis, the fist column is parameter description and the second one is parameter's $1\sigma$ uncertainty. 

In addition to GENIE parameters NOvA framework analysis uses a few more cross section effects to get a better agreement 
between ND data and simulations. First effect is RPA\footnote{Random Phase Approximation} corrections, data from MINERvA 
\cite{MINERvA} strongly support the need for a correction to reactions at low momentum transfer - QE and RES reactions. 
Corrections were needed to accommodate effect of "weak charge screening" and they were computed based on RPA. Both corrections 
and uncertainties are momentum transfer dependent except for RPA RES unsertainty which is taken at 100\% (means that effect 
is switched off). Second effect is MEC, .... 


Figures \ref{fig:MaCCQEreduced}, \ref{fig:MECEnuShapeNu}, \ref{fig:MaCCRES}, \ref{fig:MvCCRES}, \ref{fig:CCQEPauliSupViaKF},
\ref{fig:DISvnCC1pi} illustrate effect of GENIE parameters on selected escaping signal and background events. Those parameters
are choosen to be shown here as they have a noticable impact on FD predictions while the rest of the cross section parameters
have much smaller effect.

{\renewcommand{\arraystretch}{2}%
\begin{table}[!th]
\centering
\begin{tabular}{ m{10cm} m{2.5cm} }
  \hline\hline
  \text{Description} & \text{Uncertainty} \tabularnewline
  \hline
  Axial mass for CC quasi-elastic neutrino production & $\pm5\%$ \tabularnewline
  Axial mass for CC resonance neutrino production & $\pm20\%$ \tabularnewline
  Axial mass for NC resonance neutrino production & $\pm20\%$ \tabularnewline
  Vector mass for CC resonance neutrino production & $\pm10\%$ \tabularnewline
  CCQE Pauli suppression (via changes in Fermi level $k_F$ ) & $\pm35\%$ \tabularnewline
  CC coherent neutrino production, normalization & $\pm20\%$ \tabularnewline
  NC coherent neutrino production, normalization & $\pm20\%$ \tabularnewline
  Nucleon elastic reaction probability & $\pm30\%$ \tabularnewline
  Nucleon charge exchange probability & $\pm50\%$ \tabularnewline
  Nucleon absorption probability & $\pm20\%$ \tabularnewline
  Nucleon inelastic reaction probability & $\pm40\%$ \tabularnewline
  Hadron formation zone & $\pm50\%$ \tabularnewline
  DIS neutrino on neutron CC 1 pion production & $\pm50\%$ \tabularnewline
  DIS neutrino on neutron CC 2 pions production & $\pm50\%$ \tabularnewline
  \hline\hline
\end{tabular}
\caption{GENIE parameters and their uncertainties used in the analysis.}
\label{table:XSecParam}
\end{table}}

{\renewcommand{\arraystretch}{2}%
\begin{table}[th]
\centering
\begin{tabular}{ m{10cm} m{2.5cm} }
  \hline\hline
  \text{Description} & \text{Uncertainty} \tabularnewline
  \hline
  RPA effect in resonance neutrino production & $q^2$ dependent \tabularnewline
  RPA enhancement effect in CC QE & $q^2$ dependent \tabularnewline
  RPA suppresion effect in CC QE & $q^2$ dependent \tabularnewline
  MEC initial state np fraction & 0.7...0.9 \tabularnewline
  MEC cross section energy dependence & $\pm\frac{1}{1+2.5E_{\nu_\mu}}$ \tabularnewline
  MEC momentum transfer shape & $q^2$ dependent \tabularnewline
  \hline\hline
\end{tabular}
\caption{RPA and MEC uncertainties used in the analysis.}
\label{table:RPAMEC}
\end{table}}

\clearpage
\begin{figure}[t!]
\begin{subfigure}[t]{0.5\textwidth}
  \centering
  \includegraphics[width=1.0\linewidth]{figures/Syst/compare_pred_xp_numu_MaCCQEreduced2018_EscNumuEnergy_1sigma_bkg_mccomp_ratio.pdf}
  \caption{FD MC Background Prediction}
  \label{fig:bkg_MaCCQEreduced}
\end{subfigure}%
\begin{subfigure}[t]{0.5\textwidth}
  \centering
  \includegraphics[width=1.0\linewidth]{figures/Syst/compare_pred_xp_numu_MaCCQEreduced2018_EscNumuEnergy_1sigma_sig_mccomp_ratio.pdf}
  \caption{FD MC Signal Prediction}
  \label{fig:sig_MaCCQEreduced}
\end{subfigure}
\caption{ Systematic effect of MaCCQE uncertainty }
\label{fig:MaCCQEreduced}
\end{figure}

{\begin{table}[!th]\centerline {
 \begin{tabular}{ l | r r r | r r }
          Integral shift  &    Nominal &  Shift (+) &  Shift (-) & \% Diff. (+) & \% Diff. (-)   \\ \hline
       $\nu_{\mu}$ signal &         18 &         17 &         18 &       -0.7 &       +1.0   \\
             Tot beam bkg &       0.46 &       0.46 &       0.46 &       -0.6 &       +0.3   \\
                       NC &       0.33 &       0.33 &       0.33 &       -0.1 &       -0.1   \\
          $\nu_{\mu}$ App &      0.027 &      0.026 &      0.027 &       -1.6 &       +1.6   \\
             $\nu_{e}$ CC &      0.051 &      0.051 &      0.051 &       +0.0 &       +0.0   \\
          $\nu_{\tau}$ CC &      0.056 &      0.055 &      0.058 &       -2.1 &       +2.7   \\
\end{tabular} }
\caption{ Systematic effect of MaCCQE uncertainty on FD MC signal and background prediction}
{"Nominal" column shows number of selected events when GENIE parameter used at its central value, columns "Shift ($\pm$)"
show number selected events when GENIE parameter shifted by $1\sigma$ up or down correspondingly, columns "Diff ($\pm$)"
show the same shifts in a relative manner.}
\end{table} }

\newpage
\begin{figure}[t!]
\begin{subfigure}[t]{0.5\textwidth}
  \centering
  \includegraphics[width=1.0\linewidth]{figures/Syst/compare_pred_xp_numu_MECEnuShapeNu_EscNumuEnergy_1sigma_bkg_mccomp_ratio.pdf}
  \caption{FD MC Background Prediction}
  \label{fig:bkg_MECEnuShapeNu}
\end{subfigure}%
\begin{subfigure}[t]{0.5\textwidth}
  \centering
  \includegraphics[width=1.0\linewidth]{figures/Syst/compare_pred_xp_numu_MECEnuShapeNu_EscNumuEnergy_1sigma_sig_mccomp_ratio.pdf}
  \caption{FD MC Signal Prediction}
  \label{fig:sig_MECEnuShapeNu}
\end{subfigure}
\caption{ Systematic effect of MECEnuShapeNu uncertainty }
\label{fig:MECEnuShapeNu}
\end{figure}

{\begin{table}[!th]\centerline {
 \begin{tabular}{ l | r r r | r r }
          Integral shift  &    Nominal &  Shift (+) &  Shift (-) & \% Diff. (+) & \% Diff. (-)   \\ \hline
       $\nu_{\mu}$ signal &         18 &         18 &         17 &       +1.8 &       -1.4   \\
             Tot beam bkg &       0.46 &       0.46 &       0.46 &       -0.4 &       +0.0   \\
                       NC &       0.33 &       0.33 &       0.33 &       -0.1 &       -0.1   \\
          $\nu_{\mu}$ App &      0.027 &      0.026 &      0.027 &       -1.6 &       +1.6   \\
             $\nu_{e}$ CC &      0.051 &      0.051 &      0.051 &       +0.0 &       +0.0   \\
          $\nu_{\tau}$ CC &      0.056 &      0.056 &      0.057 &       -0.6 &       +1.1   \\
\end{tabular} }
\caption{ Systematic effect of MECEnuShape uncertainty on FD MC signal and background prediction}
{"Nominal" column shows number of selected events when GENIE parameter used at its central value, columns "Shift ($\pm$)"
show number selected events when GENIE parameter shifted by $1\sigma$ up or down correspondingly, columns "Diff ($\pm$)"
show the same shifts in a relative manner.}
\end{table} }

\newpage
\begin{figure}[t!]
\begin{subfigure}[t]{0.5\textwidth}
  \centering
  \includegraphics[width=1.0\linewidth]{figures/Syst/compare_pred_xp_numu_MaCCRES_EscNumuEnergy_1sigma_bkg_mccomp_ratio.pdf}
  \caption{FD MC Background Prediction}
  \label{fig:bkg_MaCCRES}
\end{subfigure}%
\begin{subfigure}[t]{0.5\textwidth}
  \centering
  \includegraphics[width=1.0\linewidth]{figures/Syst/compare_pred_xp_numu_MaCCRES_EscNumuEnergy_1sigma_sig_mccomp_ratio.pdf}
  \caption{FD MC Signal Prediction}
  \label{fig:sig_MaCCRES}
\end{subfigure}
\caption{ Systematic effect of MaCCRES uncertainty }
\label{fig:MaCCRES}
\end{figure}

{\begin{table}[!th]\centerline {
 \begin{tabular}{ l | r r r | r r }
          Integral shift  &    Nominal &  Shift (+) &  Shift (-) & \% Diff. (+) & \% Diff. (-)   \\ \hline
       $\nu_{\mu}$ signal &         18 &         17 &         19 &       -5.4 &       +8.1   \\
             Tot beam bkg &       0.46 &       0.45 &       0.48 &       -3.7 &       +4.5   \\
                       NC &       0.33 &       0.33 &       0.33 &       -0.1 &       -0.1   \\
          $\nu_{\mu}$ App &      0.027 &      0.024 &      0.031 &      -11.9 &      +13.9   \\
             $\nu_{e}$ CC &      0.051 &      0.051 &      0.051 &       +0.0 &       +0.0   \\
          $\nu_{\tau}$ CC &      0.056 &      0.048 &      0.068 &      -14.1 &      +21.3   \\
\end{tabular} }
\caption{ Systematic effect of MaCCRES uncertainty on FD MC signal and background prediction}
{"Nominal" column shows number of selected events when GENIE parameter used at its central value, columns "Shift ($\pm$)"
show number selected events when GENIE parameter shifted by $1\sigma$ up or down correspondingly, columns "Diff ($\pm$)"
show the same shifts in a relative manner.}
\end{table} }

\newpage
\begin{figure}[t!]
\begin{subfigure}[t]{0.5\textwidth}
  \centering
  \includegraphics[width=1.0\linewidth]{figures/Syst/compare_pred_xp_numu_MvCCRES_EscNumuEnergy_1sigma_bkg_mccomp_ratio.pdf}
  \caption{FD MC Background Prediction}
  \label{fig:bkg_MvCCRES}
\end{subfigure}%
\begin{subfigure}[t]{0.5\textwidth}
  \centering
  \includegraphics[width=1.0\linewidth]{figures/Syst/compare_pred_xp_numu_MvCCRES_EscNumuEnergy_1sigma_sig_mccomp_ratio.pdf}
  \caption{FD MC Signal Prediction}
  \label{fig:sig_MvCCRES}
\end{subfigure}
\caption{ Systematic effect of MvCCRES uncertainty }
\label{fig:MvCCRES}
\end{figure}

{\begin{table}[!th]\centerline {
 \begin{tabular}{ l | r r r | r r }
          Integral shift  &    Nominal &  Shift (+) &  Shift (-) & \% Diff. (+) & \% Diff. (-)   \\ \hline
       $\nu_{\mu}$ signal &         18 &         17 &         18 &       -3.2 &       +4.4   \\
             Tot beam bkg &       0.46 &       0.45 &       0.47 &       -2.1 &       +2.3   \\
                       NC &       0.33 &       0.33 &       0.33 &       -0.1 &       -0.1   \\
          $\nu_{\mu}$ App &      0.027 &      0.025 &      0.029 &       -6.3 &       +7.6   \\
             $\nu_{e}$ CC &      0.051 &      0.051 &      0.051 &       +0.0 &       +0.0   \\
          $\nu_{\tau}$ CC &      0.056 &      0.052 &      0.062 &       -7.5 &      +10.8   \\
\end{tabular} }
\caption{ Systematic effect of MvCCRES uncertainty on FD MC signal and background prediction}
{"Nominal" column shows number of selected events when GENIE parameter used at its central value, columns "Shift ($\pm$)"
show number selected events when GENIE parameter shifted by $1\sigma$ up or down correspondingly, columns "Diff ($\pm$)"
show the same shifts in a relative manner.}
\end{table} }

\newpage
\begin{figure}[t!]
\begin{subfigure}[t]{0.5\textwidth}
  \centering
  \includegraphics[width=1.0\linewidth]{figures/Syst/compare_pred_xp_numu_CCQEPauliSupViaKF_EscNumuEnergy_1sigma_bkg_mccomp_ratio.pdf}
  \caption{FD MC Background Prediction}
  \label{fig:bkg_CCQEPauliSupViaKF}
\end{subfigure}%
\begin{subfigure}[t]{0.5\textwidth}
  \centering
  \includegraphics[width=1.0\linewidth]{figures/Syst/compare_pred_xp_numu_CCQEPauliSupViaKF_EscNumuEnergy_1sigma_sig_mccomp_ratio.pdf}
  \caption{FD MC Signal Prediction}
  \label{fig:sig_CCQEPauliSupViaKF}
\end{subfigure}
\caption{ Systematic effect of CCQEPauliSupViaKF uncertainty }
\label{fig:CCQEPauliSupViaKF}
\end{figure}

{\begin{table}[!th]\centerline {
 \begin{tabular}{ l | r r r | r r }
          Integral shift  &    Nominal &  Shift (+) &  Shift (-) & \% Diff. (+) & \% Diff. (-)   \\ \hline
       $\nu_{\mu}$ signal &         18 &         17 &         18 &       -1.2 &       +1.3   \\
             Tot beam bkg &       0.46 &       0.46 &       0.46 &       -0.2 &       -0.2   \\
                       NC &       0.33 &       0.33 &       0.33 &       -0.0 &       -0.0   \\
          $\nu_{\mu}$ App &      0.027 &      0.027 &      0.027 &       -0.7 &       -0.7   \\
             $\nu_{e}$ CC &      0.051 &      0.051 &      0.051 &       +0.0 &       +0.0   \\
          $\nu_{\tau}$ CC &      0.056 &      0.056 &      0.056 &       +0.0 &       -0.0   \\
\end{tabular} }
\caption{ Systematic effect of CCQEPauliSupViaKF uncertainty on FD MC signal and background prediction}
{"Nominal" column shows number of selected events when GENIE parameter used at its central value, columns "Shift ($\pm$)"
show number selected events when GENIE parameter shifted by $1\sigma$ up or down correspondingly, columns "Diff ($\pm$)"
show the same shifts in a relative manner.}
\end{table} }

\newpage
\begin{figure}[t!]
\begin{subfigure}[t]{0.5\textwidth}
  \centering
  \includegraphics[width=1.0\linewidth]{figures/Syst/compare_pred_xp_numu_DISvnCC1pi_EscNumuEnergy_1sigma_bkg_mccomp_ratio.pdf}
  \caption{FD MC Background Prediction}
  \label{fig:bkg_DISvnCC1pi}
\end{subfigure}%
\begin{subfigure}[t]{0.5\textwidth}
  \centering
  \includegraphics[width=1.0\linewidth]{figures/Syst/compare_pred_xp_numu_DISvnCC1pi_EscNumuEnergy_1sigma_sig_mccomp_ratio.pdf}
  \caption{FD MC Signal Prediction}
  \label{fig:sig_DISvnCC1pi}
\end{subfigure}
\caption{ Systematic effect of DISvnCC1pi uncertainty }
\label{fig:DISvnCC1pi}
\end{figure}

{\begin{table}[!th]\centerline {
 \begin{tabular}{ l | r r r | r r }
          Integral shift  &    Nominal &  Shift (+) &  Shift (-) & \% Diff. (+) & \% Diff. (-)   \\ \hline
       $\nu_{\mu}$ signal &         18 &         17 &         18 &       -1.2 &       +1.3   \\
             Tot beam bkg &       0.46 &       0.46 &       0.46 &       -0.2 &       -0.2   \\
                       NC &       0.33 &       0.33 &       0.33 &       -0.0 &       -0.0   \\
          $\nu_{\mu}$ App &      0.027 &      0.027 &      0.027 &       -0.7 &       -0.7   \\
             $\nu_{e}$ CC &      0.051 &      0.051 &      0.051 &       +0.0 &       +0.0   \\
          $\nu_{\tau}$ CC &      0.056 &      0.056 &      0.056 &       +0.0 &       -0.0   \\
\end{tabular} }
\caption{ Systematic effect of DISvnCC1pi uncertainty on FD MC signal and background prediction}
{"Nominal" column shows number of selected events when GENIE parameter used at its central value, columns "Shift ($\pm$)"
show number selected events when GENIE parameter shifted by $1\sigma$ up or down correspondingly, columns "Diff ($\pm$)"
show the same shifts in a relative manner.}
\end{table} }

\section{Flux Uncertainty}
In order to estimate neutrino flux through the Near and the Far detectors the NuMI beam have to be simulated from the hit proton
hits of the target through pions, muons and kaons decay in the decay pipe. And systematic uncertainties of all these processes
have to be evaluated.

Flux uncertainties could be separated into two categories 
\begin{itemize}
\item \textbf{Hadron production}. Uncertainties on cross section of proton interaction in the NuMI target. 
\item \textbf{Beam transport}. Uncertainties on target facility components:
\begin{itemize}
	\item Horn current, 5\% uncertainty
	\item Horn 1 and Horn 2 positions in both transverse dimensions,  $\pm2$ mm uncertainties
	\item Beam position on target, $\pm0.5$ mm uncertainty
        \item Beam spot size on target, $\pm0.2$ mm uncertainty
	\item Target posistion in longitudinal direction, $\pm2$ mm uncertainty
\end{itemize} 
\end{itemize}

It turns out that flux uncertainties are heavily correlated accros almost all neutrino true energy bins. To correctly treat this,
2000 predictions were generates in different flux universes where parameters described above have different values within their
uncertainty limits, \cite{fluxPCA_technote}. Correlation matrix is constructed based on flux histograms from different universes 
and after that a Principal Component Analysis (PCA) is carried out to disintangle the correlations. It is shown in 
\cite{fluxPCA_technote} that only first 5 or 8 PCA components are enough to explain most of the information hidden in correlation 
matrix i.e. first several PCA components describes all flux uncertainties. 

As mentioned previously two detectors and extrapolation procedure help to reduce flux systematics significantly, in figure 
\ref{fig:ppfxhadpbeampc02} shown 3 PCA component which has the biggest impact on the Far detector prediction.

\newpage
\begin{figure}[t!]
\begin{subfigure}[t]{0.5\textwidth}
  \centering
  \includegraphics[width=1.0\linewidth]{figures/Syst/compare_pred_xp_numu_ppfxhadpbeampc02_EscNumuEnergy_1sigma_bkg_mccomp_ratio.pdf}
  \caption{FD MC Background Prediction}
  \label{fig:bkg_ppfxhadpbeampc02}
\end{subfigure}%
\begin{subfigure}[t]{0.5\textwidth}
  \centering
  \includegraphics[width=1.0\linewidth]{figures/Syst/compare_pred_xp_numu_ppfxhadpbeampc02_EscNumuEnergy_1sigma_sig_mccomp_ratio.pdf}
  \caption{FD MC Signal Prediction}
  \label{fig:sig_ppfxhadpbeampc02}
\end{subfigure}
\caption{ Systematic effect of 3rd flux PCA component.}
\label{fig:ppfxhadpbeampc02}
\end{figure}

{\begin{table}[!th]\centerline {
 \begin{tabular}{ l | r r r | r r }
          Integral shift  &    Nominal &  Shift (+) &  Shift (-) & \% Diff. (+) & \% Diff. (-)   \\ \hline
       $\nu_{\mu}$ signal &         18 &         17 &         18 &       -0.7 &       +0.7   \\
             Tot beam bkg &       0.46 &       0.47 &       0.46 &       +1.2 &       -1.3   \\
                       NC &       0.33 &       0.33 &       0.32 &       +1.3 &       -1.4   \\
          $\nu_{\mu}$ App &      0.027 &      0.027 &      0.027 &       -0.5 &       +0.5   \\
             $\nu_{e}$ CC &      0.051 &      0.051 &      0.051 &       +0.0 &       +0.0   \\
          $\nu_{\tau}$ CC &      0.056 &      0.057 &      0.055 &       +1.4 &       -1.6   \\
\end{tabular} }
\caption{ Systematic effect of 3rd flux PCA component on FD MC signal and background prediction}
{"Nominal" column shows number of selected events when flux parameters used at its central value, columns "Shift ($\pm$)"
show number selected events when 3rd flux PCA component shifted by $1\sigma$ up or down correspondingly, columns "Diff ($\pm$)"
show the same shifts in a relative manner.}
\end{table} } 

\section{Muon Energy Uncertainty}
Due to some uncertainties in detectors masses, compositions, offsets etc. the observed range traveled by muon has its own 
uncertainty. Muon track length is used directly to estimate muon energy at the Near detector and Far detector (for contained 
sample) and is important part of energy estimator for escaping sample at the Far Detector. A very extensive study was carried out
in \cite{muonEUnc_technote} and following absolute and relative systematic uncertainties were established
\begin{itemize}
\item \textbf{Absolute}: 
\begin{itemize}
	\item muon track length in active region of the Near and Far detectors, 0.94\%
	\item muon track length in muon catcher at the Near detector, 0.69\%
\end{itemize}
\item \textbf{Relative}:
\begin{itemize}
        \item muon track length in active region of the Near and Far detectors, 0.27\%
        \item muon track length in muon catcher at the Near detector, 0.75\%
\end{itemize}
\end{itemize}

Figures \ref{fig:AbsMuEScale2017} and \ref{fig:RelMuEScale2017} illustrate impact of absolute and relative systematic 
uncertainties on ecaping sample prediction at the Far detector correspondingly.

\clearpage
\begin{figure}[t!]
\begin{subfigure}[t]{0.5\textwidth}
  \centering
  \includegraphics[width=1.0\linewidth]{figures/Syst/compare_pred_xp_numu_AbsMuEScale2017_EscNumuEnergy_1sigma_bkg_mccomp_ratio.pdf}
  \caption{FD MC Background Prediction}
  \label{fig:bkg_AbsMuEScale2017}
\end{subfigure}%
\begin{subfigure}[t]{0.5\textwidth}
  \centering
  \includegraphics[width=1.0\linewidth]{figures/Syst/compare_pred_xp_numu_AbsMuEScale2017_EscNumuEnergy_1sigma_sig_mccomp_ratio.pdf}
  \caption{FD MC Signal Prediction}
  \label{fig:sig_AbsMuEScale2017}
\end{subfigure}
\caption{ Systematic effect of absolute muon energy uncertainty }
\label{fig:AbsMuEScale2017}
\end{figure}

{\begin{table}[!th]\centerline {
 \begin{tabular}{ l | r r r | r r }
          Integral shift  &    Nominal &  Shift (+) &  Shift (-) & \% Diff. (+) & \% Diff. (-)   \\ \hline
       $\nu_{\mu}$ signal &         18 &         17 &         18 &       -2.2 &       +2.2   \\
             Tot beam bkg &       0.46 &       0.46 &       0.46 &       -0.2 &       +0.1   \\
                       NC &       0.33 &       0.33 &       0.33 &       -0.2 &       +0.0   \\
          $\nu_{\mu}$ App &      0.027 &      0.027 &      0.027 &       -0.4 &       +0.3   \\
             $\nu_{e}$ CC &      0.051 &      0.051 &      0.051 &       +0.0 &       +0.0   \\
          $\nu_{\tau}$ CC &      0.056 &      0.056 &      0.056 &       -0.6 &       +0.4   \\
\end{tabular} }
\caption{ Systematic effect of absolute muon energy uncertainty on FD MC signal and background prediction}
{"Nominal" column shows number of selected events for nominal simulations, columns "Shift ($\pm$)"
show number selected events when muon track length shifted by $1\sigma$ up or down correspondingly, columns "Diff ($\pm$)"
show the same shifts in a relative manner.}
\end{table} }

\newpage
\begin{figure}[t!]
\begin{subfigure}[t]{0.5\textwidth}
  \centering
  \includegraphics[width=1.0\linewidth]{figures/Syst/compare_pred_xp_numu_RelMuEScale2017_EscNumuEnergy_1sigma_bkg_mccomp_ratio.pdf}
  \caption{FD MC Background Prediction}
  \label{fig:bkg_RelMuEScale2017}
\end{subfigure}%
\begin{subfigure}[t]{0.5\textwidth}
  \centering
  \includegraphics[width=1.0\linewidth]{figures/Syst/compare_pred_xp_numu_RelMuEScale2017_EscNumuEnergy_1sigma_sig_mccomp_ratio.pdf}
  \caption{FD MC Signal Prediction}
  \label{fig:sig_RelMuEScale2017}
\end{subfigure}
\caption{ Systematic effect of relative muon energy uncertainty }
\label{fig:RelMuEScale2017}
\end{figure}

{\begin{table}[!th]\centerline {
 \begin{tabular}{ l | r r r | r r }
          Integral shift  &    Nominal &  Shift (+) &  Shift (-) & \% Diff. (+) & \% Diff. (-)   \\ \hline
       $\nu_{\mu}$ signal &         18 &         17 &         18 &       -0.5 &       +0.4   \\
             Tot beam bkg &       0.46 &       0.46 &       0.46 &       +0.0 &       -0.0   \\
                       NC &       0.33 &       0.33 &       0.33 &       +0.0 &       +0.0   \\
          $\nu_{\mu}$ App &      0.027 &      0.027 &      0.027 &       +0.1 &       -0.1   \\
             $\nu_{e}$ CC &      0.051 &      0.051 &      0.051 &       +0.0 &       +0.0   \\
          $\nu_{\tau}$ CC &      0.056 &      0.056 &      0.056 &       +0.0 &       -0.1   \\
\end{tabular} }
\caption{ Systematic effect of relative muon energy uncertainty on FD MC signal and background prediction}
{"Nominal" column shows number of selected events for nominal simulations, columns "Shift ($\pm$)"
show number selected events when muon track length shifted by $1\sigma$ up or down correspondingly, columns "Diff ($\pm$)"
show the same shifts in a relative manner.}
\end{table} }

%\section{Hadronic Energy Uncertainty}
%Hadronic energy uncertainty is 5\% both absolute and relative.
%
%\clearpage
%\begin{figure}[t!]
%\begin{subfigure}[t]{0.5\textwidth}
%  \centering
%  \includegraphics[width=1.0\linewidth]{figures/Syst/compare_pred_xp_numu_AbsHadEScale2017_EscNumuEnergy_1sigma_bkg_mccomp_ratio.pdf}
%  \caption{FD MC Background Prediction}
%  \label{fig:bkg_AbsHadEScale2017}
%\end{subfigure}%
%\begin{subfigure}[t]{0.5\textwidth}
%  \centering
%  \includegraphics[width=1.0\linewidth]{figures/Syst/compare_pred_xp_numu_AbsHadEScale2017_EscNumuEnergy_1sigma_sig_mccomp_ratio.pdf}
%  \caption{FD MC Signal Prediction}
%  \label{fig:sig_AbsHadEScale2017}
%\end{subfigure}
%\caption{ Systematic effect of absolute hadronic energy uncertainty }
%\label{fig:AbsHadEScale2017}
%\end{figure}
%
%{\begin{table}[!th]\centerline {
% \begin{tabular}{ l | r r r | r r }
%          Integral shift  &    Nominal &  Shift (+) &  Shift (-) & \% Diff. (+) & \% Diff. (-)   \\ \hline
%       $\nu_{\mu}$ signal &         18 &         19 &         17 &       +5.8 &       -5.7   \\
%             Tot beam bkg &       0.46 &       0.47 &       0.46 &       +1.5 &       -1.4   \\
%                       NC &       0.33 &       0.34 &       0.32 &       +1.8 &       -1.4   \\
%          $\nu_{\mu}$ App &      0.027 &      0.027 &      0.027 &       +0.4 &       -1.0   \\
%             $\nu_{e}$ CC &      0.051 &      0.051 &      0.051 &       +0.0 &       +0.0   \\
%          $\nu_{\tau}$ CC &      0.056 &      0.057 &      0.055 &       +1.5 &       -1.8   \\
%\end{tabular} }
%\caption{ Systematic effect of absolute hadronic energy uncertainty on FD MC signal and background prediction}
%{"Nominal" column shows number of selected events for nominal simulations, columns "Shift ($\pm$)"
%show number selected events when hadronic energy shifted by $1\sigma$ up or down correspondingly, columns "Diff ($\pm$)"
%show the same shifts in a relative manner.}
%\end{table} }
%
%\newpage
%\begin{figure}[t!]
%\begin{subfigure}[t]{0.5\textwidth}
%  \centering
%  \includegraphics[width=1.0\linewidth]{figures/Syst/compare_pred_xp_numu_RelHadEScale2017_EscNumuEnergy_1sigma_bkg_mccomp_ratio.pdf}
%  \caption{FD MC Background Prediction}
%  \label{fig:bkg_RelHadEScale2017}
%\end{subfigure}%
%\begin{subfigure}[t]{0.5\textwidth}
%  \centering
%  \includegraphics[width=1.0\linewidth]{figures/Syst/compare_pred_xp_numu_RelHadEScale2017_EscNumuEnergy_1sigma_sig_mccomp_ratio.pdf}
%  \caption{FD MC Signal Prediction}
%  \label{fig:sig_RelHadEScale2017}
%\end{subfigure}
%\caption{ Systematic effect of relative hadronic energy uncertainty }
%\label{fig:RelHadEScale2017}
%\end{figure}
%
%{\begin{table}[!th]\centerline {
% \begin{tabular}{ l | r r r | r r }
%          Integral shift  &    Nominal &  Shift (+) &  Shift (-) & \% Diff. (+) & \% Diff. (-)   \\ \hline
%       $\nu_{\mu}$ signal &         18 &         18 &         17 &       +1.3 &       -1.4   \\
%             Tot beam bkg &       0.46 &       0.46 &       0.47 &       -0.9 &       +0.6   \\
%                       NC &       0.33 &       0.33 &       0.33 &       -0.8 &       +0.7   \\
%          $\nu_{\mu}$ App &      0.027 &      0.027 &      0.027 &       -0.5 &       +0.1   \\
%             $\nu_{e}$ CC &      0.051 &      0.051 &      0.051 &       +0.0 &       +0.0   \\
%          $\nu_{\tau}$ CC &      0.056 &      0.056 &      0.057 &       -0.9 &       +0.8   \\
%\end{tabular} }
%\caption{ Systematic effect of relative hadronic energy uncertainty on FD MC signal and background prediction}
%{"Nominal" column shows number of selected events for nominal simulations, columns "Shift ($\pm$)"
%show number selected events when hadronic energy shifted by $1\sigma$ up or down correspondingly, columns "Diff ($\pm$)"
%show the same shifts in a relative manner.}
%\end{table} }

\section{Calorimetric Energy Uncertainties}
Several studies of post calibration calorimetric response of the detectors show that the protons in the Near detector
have the biggest discrepancy between data and simulations \cite{ecalib_talk}. It is found that calorimetric detector
response in data is 5\% lower than the simulation and this desrepancy is interpreted as a 5\% absolute calibration
uncertainty. In order to evaluate an impact of this uncertainty a new simulation is generated and a different calibration
is applied before the reconstruction step. As result, every slice has the same number of hits but their energies are 
shifted up or down. Figure \ref{fig:Calibration} illustrates the impact of absolute calibration uncertainty.

While comparison of simulated muons to data can be done in both detectors (simulation/data comparison in the Near detector 
and simulation/cosmic comparison in the Far detector) however for protons it could be done only in the Near detector.  
Thus, the same 5\% effect is assumed as the worst case for a relative effect. Figure \ref{fig:RelativeCalib} illustrates 
impact of relative calibration uncertainty on selected escaping sample at the Far detector. 

A detector cell has a different response for energy deposition at different distances to the readout electronics. 
A comparison of the reconstructed and true energy of simulated cosmic hits as function of their position along the cell 
is performed in \cite{shape_talk} and the discrepancy is interpreted as calibration 'shape' uncertainty. The effect of this 
systematic is illustrated in figure \ref{fig:CalibShape}.  

\clearpage
\begin{figure}[t!]
\begin{subfigure}[t]{0.5\textwidth}
  \centering
  \includegraphics[width=1.0\linewidth]{figures/Syst/compare_pred_xp_numu_Calibration_EscNumuEnergy_1sigma_bkg_mccomp_ratio.pdf}
  \caption{FD MC Background Prediction}
  \label{fig:bkg_Calibration}
\end{subfigure}%
\begin{subfigure}[t]{0.5\textwidth}
  \centering
  \includegraphics[width=1.0\linewidth]{figures/Syst/compare_pred_xp_numu_Calibration_EscNumuEnergy_1sigma_sig_mccomp_ratio.pdf}
  \caption{FD MC Signal Prediction}
  \label{fig:sig_Calibration}
\end{subfigure}
\caption{ Systematic effect of absolute calibration uncertainty }
\label{fig:Calibration}
\end{figure}

{\begin{table}[!th]\centerline {
 \begin{tabular}{ l | r r r | r r }
          Integral shift  &    Nominal &  Shift (+) &  Shift (-) & \% Diff. (+) & \% Diff. (-)   \\ \hline
       $\nu_{\mu}$ signal &         18 &         19 &         16 &       +8.5 &      -10.1   \\
             Tot beam bkg &       0.46 &       0.54 &       0.39 &      +17.4 &      -15.6   \\
                       NC &       0.33 &       0.41 &       0.29 &      +23.5 &      -13.2   \\
          $\nu_{\mu}$ App &      0.027 &      0.026 &      0.023 &       -2.0 &      -12.9   \\
             $\nu_{e}$ CC &      0.051 &      0.051 &      0.051 &       +0.0 &       +0.0   \\
          $\nu_{\tau}$ CC &      0.056 &      0.057 &      0.048 &       +2.1 &      -14.9   \\
\end{tabular} }
\caption{ Systematic effect of absolute calibration uncertainty on FD MC signal and background prediction}
{"Nominal" column shows number of selected events for nominal simulations, columns "Shift ($\pm$)"
show number selected events when calibration shifted sample is used, columns "Diff ($\pm$)" show the same shifts 
in a relative manner.}
\end{table} }

\newpage
\begin{figure}[t!]
\begin{subfigure}[t]{0.5\textwidth}
  \centering
  \includegraphics[width=1.0\linewidth]{figures/Syst/compare_pred_xp_numu_RelativeCalib_EscNumuEnergy_1sigma_bkg_mccomp_ratio.pdf}
  \caption{FD MC Background Prediction}
  \label{fig:bkg_RelativeCalib}
\end{subfigure}%
\begin{subfigure}[t]{0.5\textwidth}
  \centering
  \includegraphics[width=1.0\linewidth]{figures/Syst/compare_pred_xp_numu_RelativeCalib_EscNumuEnergy_1sigma_sig_mccomp_ratio.pdf}
  \caption{FD MC Signal Prediction}
  \label{fig:sig_RelativeCalib}
\end{subfigure}
\caption{ Systematic effect of relative muon energy uncertainty }
\label{fig:RelativeCalib}
\end{figure}

{\begin{table}[!th]\centerline {
 \begin{tabular}{ l | r r r | r r }
          Integral shift  &    Nominal &  Shift (+) &  Shift (-) & \% Diff. (+) & \% Diff. (-)   \\ \hline
       $\nu_{\mu}$ signal &         18 &         20 &         15 &      +12.2 &      -13.5   \\
             Tot beam bkg &       0.46 &       0.54 &       0.39 &      +17.4 &      -15.3   \\
                       NC &       0.33 &       0.41 &       0.29 &      +23.5 &      -13.0   \\
          $\nu_{\mu}$ App &      0.027 &      0.026 &      0.023 &       -2.0 &      -12.8   \\
             $\nu_{e}$ CC &      0.051 &      0.051 &      0.051 &       +0.0 &       +0.0   \\
          $\nu_{\tau}$ CC &      0.056 &      0.057 &      0.048 &       +2.1 &      -14.3   \\
\end{tabular} }
\caption{ Systematic effect of relative calibration uncertainty on FD MC signal and background prediction}
{"Nominal" column shows number of selected events for nominal simulations, columns "Shift ($\pm$)"
show number selected events when calibration shifted sample is used, columns "Diff ($\pm$)" show the same shifts
in a relative manner.}
\end{table} }

\newpage
\begin{figure}[t!]
\begin{subfigure}[t]{0.5\textwidth}
  \centering
  \includegraphics[width=1.0\linewidth]{figures/Syst/compare_pred_xp_numu_CalibShape_EscNumuEnergy_1sigma_bkg_mccomp_ratio.pdf}
  \caption{FD MC Background Prediction}
  \label{fig:bkg_CalibShape}
\end{subfigure}%
\begin{subfigure}[t]{0.5\textwidth}
  \centering
  \includegraphics[width=1.0\linewidth]{figures/Syst/compare_pred_xp_numu_CalibShape_EscNumuEnergy_1sigma_sig_mccomp_ratio.pdf}
  \caption{FD MC Signal Prediction}
  \label{fig:sig_CalibShape}
\end{subfigure}
\caption{ Systematic effect of 'shape' calibration uncertainty }
\label{fig:CalibShape}
\end{figure}

{\begin{table}[!th]\centerline {
 \begin{tabular}{ l | r r r | r r }
          Integral shift  &    Nominal &  Shift (+) &  Shift (-) & \% Diff. (+) & \% Diff. (-)   \\ \hline
  $\nu_{\mu}$ signal &     17.524 &     15.811 &         -- &       -9.8 &         -- \\
        Tot beam bkg &      0.463 &      0.428 &         -- &       -7.5 &         -- \\
        $\nu_{e}$ CC &      0.051 &      0.040 &         -- &      -20.2 &         -- \\
                  NC &      0.329 &      0.315 &         -- &       -4.2 &         -- \\
     $\nu_{\mu} $App &      0.027 &      0.023 &         -- &      -14.2 &         -- \\
     $\nu_{\tau}$ CC &      0.056 &      0.049 &         -- &      -12.5 &         -- \\
\end{tabular} }
\caption{ Systematic effect of 'shape' calibration uncertainty on FD MC signal and background prediction}
{"Nominal" column shows number of selected events for nominal simulations, columns "Shift ($\pm$)"
show number selected events when calibration shifted sample is used, columns "Diff ($\pm$)" show the same shifts
in a relative manner.}
\end{table} }

%%%%%%%%%%%%%%%%%%%%%%%%%%%%%%%%%%%%%%%%%%%%%%%%%%%%%%%%%%%%%%%%%%%%%%%%%%%%%%%%
