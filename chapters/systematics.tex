%%%%%%%%%%%%%%%%%%%%%%%%%%%%%%%%%%%%%%%%%%%%%%%%%%%%%%%%%%%%%%%%%%%%%%%%%%%%%%%%
% systematics.tex:
%%%%%%%%%%%%%%%%%%%%%%%%%%%%%%%%%%%%%%%%%%%%%%%%%%%%%%%%%%%%%%%%%%%%%%%%%%%%%%%%
\chapter{Systematics}
\label{systematic_chapter}
%%%%%%%%%%%%%%%%%%%%%%%%%%%%%%%%%%%%%%%%%%%%%%%%%%%%%%%%%%%%%%%%%%%%%%%%%%%%%%%%

Precise measurements of the oscillation parameters depend on correct extrapolated predictions of neutrino spectrum at 
the Far Detector. In turn, extrapolated predictions depend on our understanding of all properties of the detectors as
well as properties of particles propagation and interaction with other particles. Naturally this understanding cannot 
be absolute and the lack of it leads to uncertainties in the final result, this uncertainties are called systematic 
unsertainty, as opposed to statistical uncertainties which comes from probabalistic nature of high energy experiment.
While statistical uncertainties are mitigated by longer exposure, systematic uncertainties could be mitigated by the
experiment design. For example, two detectors design of NOvA reduces an impact of flux and cross section uncertainties.

The first section of this chapter describes the treatment of systemtic uncertainties in NOvA and the following ones
discuss the impact of different systematic uncertainties on extrapolated predictions. 

\section{Evaluation of Systematic Uncertainties}
For some systematics analysis framework allows a reweightable approach to evaluate an impact of the specific systematic
on the analysis result. This approach helps to reduce computational resources \cite{cafana}. For instance, any cross section 
parameter has it own uncertainty and for every simulated event a weight is calculated based on that uncertainty and 
truth information or the output of reconstructed algorithms, thus one gets a new (shifted) prediction which 
corresponds to that cross section parameter. One gets 2 shifted predictions by using weights which are calculated for 
$\pm 1\sigma$ shifts for some cross section parameter; nominal and 2 shifted predictions are used to produced a 
systematic error band on all plots which are presented in this chapter. Moreover, for the sake of space systematic 
error bands are shown only for escaping sample, study of systematic uncertainties for contained sample can be found in
\cite{Luke}.

Those systematic effects which need to alter information on hit-by-hit basis cannot be used in reweightable scheme 
and a completely new set of Monte Carlo simulation files is generated, for instance, new sets of files are requered for 
systematically shifted scintillation model, calibration systematics etc. 

During the fitting procedure \ref{chi2syst} fitter needs to know prediction for fractional systetamic shift. This is 
achieved by using a cubic spline interpolation between nominal, $\pm 1\sigma$ and $\pm 2\sigma$ shifts. 

Analysis framework distincts two types of systematic uncertainties in NOvA, those which are the same or correlated for 
both detectors and those which are independent. Correlated systematic uncertainties shift prediction in both detectors
simultaniously by the same amount and are called \textit{absolute} uncertainties. Flux and neutrino cross section 
uncertainties are examples of absolute uncertainties. Systematic effects which are independent across detectors 
shift prediction by the same amount but in opposite direction and are called \textit{relative} uncertainties.  

\section{Cross Section Unsertainty}
GENIE cross section models depends on 80+ parameters and uncertaintes on all these paramters allow to calculate cross
section weights. When one alters some parameter weight is determined as ratio of cross section with altered parameter
to nominal one. Later, at analysis time, every selected event at the Near or the Far Detector is weighted with above
calculated weight, thus shifted prediction is generated. The marginalization process\footnote{fitting procedure with 
systematic uncrtainties taken into account.} is computationally heavy and only those GENIE cross section systematics are 
used which has significant impact on the analysis. Table \ref{table:XSecParam} summarize cross section parameters used in
the analysis, the fist column is parameter description and the second one is parameter's $1\sigma$ uncertainty. 

{\renewcommand{\arraystretch}{2}%
\begin{table}[!th]
\centering
\begin{tabular}{ m{10cm} m{2.5cm} }
  \hline\hline
  \text{Description} & \text{Uncertainty} \tabularnewline
  \hline
  Axial mass for CC resonance neutrino production & $\pm20\%$ \tabularnewline
  Axial mass for NC resonance neutrino production & $\pm20\%$ \tabularnewline
  Vector mass for CC resonance neutrino production & $\pm10\%$ \tabularnewline
  CCQE Pauli suppression (via changes in Fermi level $k_F$ ) & $\pm35\%$ \tabularnewline
  CC coherent neutrino production, normalization & $\pm20\%$ \tabularnewline
  NC coherent neutrino production, normalization & $\pm20\%$ \tabularnewline
  RPA effect in resonance neutrino production & $q^2$ dependent \tabularnewline
  RPA enhancement effect in CC QE & $q^2$ dependent \tabularnewline
  RPA suppresion effect in CC QE & $q^2$ dependent \tabularnewline
  Nucleon elastic reaction probability & $\pm30\%$ \tabularnewline
  Nucleon charge exchange probability & $\pm50\%$ \tabularnewline
  Nucleon absorption probability & $\pm20\%$ \tabularnewline
  Nucleon inelastic reaction probability & $\pm40\%$ \tabularnewline
  Hadron formation zone & $\pm50\%$ \tabularnewline
  DIS neutrino on neutron CC 1 pion production & $\pm50\%$ \tabularnewline
  DIS neutrino on neutron CC 2 pions production & $\pm50\%$ \tabularnewline
  \hline\hline
\end{tabular}
\caption{GENIE parameters and their uncertainties used in the analysis.}
\label{table:XSecParam}
\end{table}}

Figures \ref{}, ... \ref{} illustrate effect of GENIE parameters on selected escaping signal and background events.
\newpage
\begin{figure}[t!]
\begin{subfigure}[t]{0.5\textwidth}
  \centering
  \includegraphics[width=1.0\linewidth]{figures/Syst/compare_pred_xp_numu_MaCCQEreduced2018_EscNumuEnergy_1sigma_bkg_mccomp_ratio.pdf}
  \caption{FD MC Background Prediction}
  \label{fig:bkg_MaCCQEreduced}
\end{subfigure}%
\begin{subfigure}[t]{0.5\textwidth}
  \centering
  \includegraphics[width=1.0\linewidth]{figures/Syst/compare_pred_xp_numu_MaCCQEreduced2018_EscNumuEnergy_1sigma_sig_mccomp_ratio.pdf}
  \caption{FD MC Signal Prediction}
  \label{fig:sig_MaCCQEreduced}
\end{subfigure}
\caption{ Systematic effect of MaCCQE uncertainty }
\label{fig:MaCCQEreduced}
\end{figure}

{\begin{table}[!th]\centerline {
 \begin{tabular}{ l | r r r | r r }
          Integral shift  &    Nominal &  Shift (+) &  Shift (-) & \% Diff. (+) & \% Diff. (-)   \\ \hline
       $\nu_{\mu}$ signal &         18 &         17 &         18 &       -0.7 &       +1.0   \\
             Tot beam bkg &       0.46 &       0.46 &       0.46 &       -0.6 &       +0.3   \\
                       NC &       0.33 &       0.33 &       0.33 &       -0.1 &       -0.1   \\
          $\nu_{\mu}$ App &      0.027 &      0.026 &      0.027 &       -1.6 &       +1.6   \\
             $\nu_{e}$ CC &      0.051 &      0.051 &      0.051 &       +0.0 &       +0.0   \\
          $\nu_{\tau}$ CC &      0.056 &      0.055 &      0.058 &       -2.1 &       +2.7   \\
\end{tabular} }
\caption{ Systematic effect of MaCCQE uncertainty on FD MC signal and background prediction}
{"Nominal" column shows number of selected events when GENIE parameter used at its central value, columns "Shift ($\pm$)"
show number selected events when GENIE parameter shifted by $1\sigma$ up or down correspondingly, columns "Diff ($\pm$)"
show the same shifts in relative manner.}
\end{table} }

%%%%%%%%%%%%%%%%%%%%%%%%%%%%%%%%%%%%%%%%%%%%%%%%%%%%%%%%%%%%%%%%%%%%%%%%%%%%%%%%
