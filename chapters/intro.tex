%%%%%%%%%%%%%%%%%%%%%%%%%%%%%%%%%%%%%%%%%%%%%%%%%%%%%%%%%%%%%%%%%%%%%%%%%%%%%%%
% intro.tex: Introduction to the thesis
%%%%%%%%%%%%%%%%%%%%%%%%%%%%%%%%%%%%%%%%%%%%%%%%%%%%%%%%%%%%%%%%%%%%%%%%%%%%%%%%
\chapter{Introduction}
\label{intro_chapter}
%%%%%%%%%%%%%%%%%%%%%%%%%%%%%%%%%%%%%%%%%%%%%%%%%%%%%%%%%%%%%%%%%%%%%%%%%%%%%%%%

The goal of this work is to study and understand properties of one of the most 
mysterious elementary particles in the universe - neutrinos. Neutrinos are electrically
neutral, and are lighter than all other fundamental particles by many orders of magnitude.
It is known that neutrinos could interact only through weak and 
gravitational forces which leads to a several light years of mean free path in 
the matter for 1 MeV neutrinos. Despite their tiny mass and weak interactions
with matter, they are very interesting to the elementary particle physics, cosmology, and
astrophysics communities. Neutrinos carry away almost all energy of supernova 
explosion, and can reveal a lot about a dying star's internal structure and the last 
seconds of its life. Neutrino telescopes may open a new window for Universe observation 
and give a unique look into phenomena which could not be seen by conventional 
telescopes. Neutrinos also can direct physicists
towards potential signs of new physics that can be investigated at collider experiments
\cite{seeSawAndParityViolation} and in the galaxy. The list of examples why neutrinos are 
important continues to grow, and it gives a sense that neutrino physics is a 
very rich and interesting subject to study.\\

A hallmark of the weak force is oscillations between flavors.  Oscillations were first
discovered in neutral mesons, and later in neutrinos \cite{kamiokandeTwo, solarNuSummary,
dayaBayResults, NOvAresults, mainzPhaseIIResults, t2kResults}. After travelling some
distance, neutrinos produced in one flavor oscillate into a different flavor, and this repeats
until a neutrino emits a charged weak boson and becomes a charged lepton. The NOvA - 
NuMI\footnote{Neutrino at the Main Injector} Off-axis Neutrino Appearance - experiment
was designed to study properties of these oscillations. There are several
parameters which fully describe neutrino oscillations. Analysis presented in this work 
is focused on measuring $\sin^2\theta_{23}$ and $|\Delta m^2_{32}|$. The measurements 
come from studying a specific mode of the oscillations: $\nu_\mu \rightarrow \nu_\mu$.\\

Here is a brief outline of the work

\begin{itemize}

\item Chapter 2 briefly presents the history of neutrino discovery and observation 
of neutrino oscillations.

\item Chapter 3 discusses corresponding theory behind the neutrino oscillation physics
together with the current measurements of the parameters describing these oscillations.

\item Chapter 4 describes the NOvA experiment design and its two most significant parts -
Far and Near Detectors.

\item Chapter 5 provides details on Monte Carlo simulations used in the analysis.

\item Chapter 6 follows the reconstruction chain from raw data to high level objects 
such as slices, tracks, etc. Energy estimation algorithm is considered.

\item Chapter 7 presents selection criteria for the contained sample as well as selection 
and background reduction processes for the escaping sample excluded in previous analyses.

\item Chapter 8 discusses different approaches to escaping sample energy estimation 
algorithm.

\item Chapter 9 gives predictions for the selected events in Far Detector

\item In Chapter 10 all the systematics are discussed.

\item Chapter 11 presents results of the analysis developed in the
thesis.

\item Chapter 12 summarizes the work presented in the thesis.

\end{itemize}
%%%%%%%%%%%%%%%%%%%%%%%%%%%%%%%%%%%%%%%%%%%%%%%%%%%%%%%%%%%%%%%%%%%%%%%%%%%%%%%%
