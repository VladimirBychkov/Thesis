%%%%%%%%%%%%%%%%%%%%%%%%%%%%%%%%%%%%%%%%%%%%%%%%%%%%%%%%%%%%%%%%%%%%%%%%%%%%%%%
% intro.tex: Introduction to the thesis
%%%%%%%%%%%%%%%%%%%%%%%%%%%%%%%%%%%%%%%%%%%%%%%%%%%%%%%%%%%%%%%%%%%%%%%%%%%%%%%%
\chapter{Introduction}
\label{intro_chapter}
%%%%%%%%%%%%%%%%%%%%%%%%%%%%%%%%%%%%%%%%%%%%%%%%%%%%%%%%%%%%%%%%%%%%%%%%%%%%%%%%

The goal of this work is to study and understand properties of one of the most 
mysterious elementry particles in the universe - neutrinos. Neutrino is chargeless and 
has mass which is less by many orders of magnitude than masses of others fundamental 
particles. It is known that neutrinos could interact only through weak and 
gravitational forces which leads to a several light years of mean free path in 
the matter. Despite their tiny mass and weak interaction with the matter, they are 
very interesting not only to elementary particle physics community but also to 
cosmologists and astrophysicists. Neutrinos carry away almost all energy of supernova 
explosion and can tell a lot about a dying star's internal structure and the last 
seconds of its life. Neutrino telescopes may open a new window for Universe observation 
and give a unique look into phenomena which could not be seen by conventional 
telescopes. These light, electrically neutral particles also can show physicists 
a possible direction towards new physics. The list of examples why neutrinos are 
important continues to grow, and it gives a sense that neutrino physics is a 
very rich and interesting subject to study.\\

In addition, neutrinos take part in one more fascinating phenomenon called neutrino
oscillations. While produced in one flavour after travel some distance neutrino is 
changing its flavour to another one in an oscillatory way. The NOvA - 
NuMI\footnote{Neutrino at the Main Injector} Off-axis Neutrino Appearnace - experiment
was designed to study properties of these oscillations. There are several
parameters which fully describe neutrino oscillations and analysis presented in this work 
is focused on measuring $\sin^2\theta_{23}$ and $|\Delta m^2_{32}|$. The measurements 
come from studying a specific mode of the oscillations: $\nu_\mu \rightarrow \nu_\mu$.\\

Here is a brief outline of the work

\begin{itemize}

\item Chapter 2 briefly presents the history of neutrino discovery and observation 
of neutrino oscillations.

\item Chapter 3 discusses corresponding theory behind the neutrino oscillation physics
together with the current measurements of the parameters describing these oscillations.

\item Chapter 4 describes the NOvA experiment design and its two most significant parts -
Far and Near Detectors.

\item Chapter 5 provides details on Monte Carlo simulations used in the analysis.

\item Chapter 6 follows the reconstruction chain from raw data to high level objects 
such as slices, tracks etc. Energy estimation algorithm is considered.

\item Chapter 7 ....  hashes out the strategy for analysis and presents the data and
simulated sets that will be used in the analysis.

\item Chapter 8 discusses event selection and background reduction processes for the 
contained and uncontained samples.

\item Chapter 9 gives predictions for selected events in Far Detector

\item In Chapter 10 all the systematics are discussed.

\item Chapter 11 presents a final discussion of the analyses presented in the
thesis.

\end{itemize}
%%%%%%%%%%%%%%%%%%%%%%%%%%%%%%%%%%%%%%%%%%%%%%%%%%%%%%%%%%%%%%%%%%%%%%%%%%%%%%%%
