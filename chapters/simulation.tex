%%%%%%%%%%%%%%%%%%%%%%%%%%%%%%%%%%%%%%%%%%%%%%%%%%%%%%%%%%%%%%%%%%%%%%%%%%%%%%%%
% simulation.tex: Chapter on MC production:
%%%%%%%%%%%%%%%%%%%%%%%%%%%%%%%%%%%%%%%%%%%%%%%%%%%%%%%%%%%%%%%%%%%%%%%%%%%%%%%%
\chapter{Simulation}
\label{simulation_chapter}
%%%%%%%%%%%%%%%%%%%%%%%%%%%%%%%%%%%%%%%%%%%%%%%%%%%%%%%%%%%%%%%%%%%%%%%%%%%%%%%%

%%%%%%%%%%%%%%%%%%%%%%%%%%%%%%%%%%%%%%%%%%%%%%%%%%%%%%%%%%%%%%%%%%%%%%%%%%%%%%%%
In order to predict spectra at the near and the far detectors to compare predictions with
actual data one need to simulate neutrino interactions in the detectors relying on
known physics models and theories of how particles are produced, how they travel and interact
with detector material. The simulation process in the NOvA starts with protons hiting
NuMI target and finishes with APD readouts and analog to digital converters. Side simulation
packages or custom NOvA software are used in every step, output information on every step is
used as input data for the following step
\begin{itemize}
\item Beam simulations. FLUGG simulation package which is the combination of FLUKA and GEANT4 packages.
Geometries of the target hall and detectors are encoded with the help of GEANT4, while proton
interactions and downstream particle decays are simulated with FLUKA.
\item Neutrino interactions. Neutrino flux degenrated in the previous step is an input for
the GENIE package, which performs simulation of neutrino interactions with nucleus inside the
detectors. GENIE produces a list of particles leaving the nucleus.
\item Propagation of particles through detectors. List of the particles and detailed detectors geometry are
used by GEANT4 to simulate particles propagation through the detectors. Amount of energy deposited
by the particles in every cell is the input for the last step.
\item Electronic signal. Deposited energy in the cell is converted to a light by the scintillator
which travels through the fiber to an APD and analog to digital converter. This step is
simulated by NOvA custom software.
\end{itemize}
All these 4 parts will be briefly discussed in the current chapter.

\section{Beam Simulation}
Beam simulation provides NOvA experiment with neutrino flux prediction at both detectors. 
This simulation is done by FLUKA and GEANT4 through the FLUGG interface. Detailed geometry of
the target hall, target itself, collimator, two focusing horns and decay pipe is described by
GEANT4 format files and simuation is done by FLUKA. The FLUKA package simulates 120 GeV protons
which hit the target and give rise to hadronic showes, secondary particles focused by the horns 
and enter decay pipe where they further decay and some of them produce neutrinos. All the proton 
interaction information whose dauther particles produce neutrino is saved providing a way to 
study beam uncertanties related to particle production models. 

The simulation outputs files which describe neutrino flux in terms of neutrino parent particles, 
flavour, energy and direction of motion.

\section{Simulation of Neutrino Interactions}
In order to simulate neutrino interactions in the near and far detector the GENIE simulation 
package is used. The package is developed by the experimental physics community and serves as a 
primary neutrino Monte Carlo generator for neutrino experiments due to its ability to simulate neutrino
interactions on almost any target and in wide energy region which spans from MeV to PeV. As mentioned 
above GENIE gets a result of neutrino flux simulation from a previous step and convolute it with
neutrino interaction cross sections.

The interesting feature of the neutrino energy spectrum in NOvA experiment is that it overlaps 
with energy regions of several neutrino interaction models. All these models are implemented in GENIE
simulation package. Quasi-Elastic scattering (QE), deep-inelastic scattering (DIS), baryon resonance 
production (RES) as well as meson exchange current (MEC) which dominates in 2 particle - 2 hole (2p-2h) 
effect\cite{MEC}. The physics behind the models is complicated, however on the qualitative level QE interaction 
means scattering off a single nucleon, MEC interaction means scattering off a pair of nucleons, RES process
results in excitation of the whole nucleus and DIS interaction of neutrino with a nucleus can lead to a 
complete disintegration of the nucleus\footnote{This happens when energy transferred to a nucleus is 
sufficiently large}.
