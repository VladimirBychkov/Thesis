%%%%%%%%%%%%%%%%%%%%%%%%%%%%%%%%%%%%%%%%%%%%%%%%%%%%%%%%%%%%%%%%%%%%%%%%%%%%%%%%
% reconstruction.tex:
%%%%%%%%%%%%%%%%%%%%%%%%%%%%%%%%%%%%%%%%%%%%%%%%%%%%%%%%%%%%%%%%%%%%%%%%%%%%%%%%
\chapter{Event Reconstruction}
\label{reconstruction_chapter}
%%%%%%%%%%%%%%%%%%%%%%%%%%%%%%%%%%%%%%%%%%%%%%%%%%%%%%%%%%%%%%%%%%%%%%%%%%%%%%%%

%%%%%%%%%%%%%%%%%%%%%%%%%%%%%%%%%%%%%%%%%%%%%%%%%%%%%%%%%%%%%%%%%%%%%%%%%%%%%%%%
In order to perform neutrino oscillation analysis one need to select neutrino events 
and carefully measure its energies. For this task data from the detectors have to be 
reconstructed and $\nu_\mu$ charge current (CC) interactions should be identified. The following
points briefly summarize the reconstruction chain 
\begin{itemize}
\item Slicing. The full 550$\mu s$ readout window contains hits which are associated with
neutrino interaction, cosmis rays activity and noise. The hits need to be separated into
different groups which ideally correspond to different physical processes.
\item Tracking. Group of hits ("slice") from a previous step processed by tracking algorithm
in order to detrmine 3-dimensional trajectories of particles.
\item Calibration. Before energy of the event could be estimated signal from every hit is needed 
to be converted into energy units.
\item Muon identification and Background rejection. $\nu_\mu$ CC interaction produce a muon and, 
thus finding a muon track in a slice indicates that the slice contains information about neutrino.
\item Energy estimation. Templates are used to map muon track length to true muon energy and visible 
hadronic energy\footnote{Total energy of all hits in the slice which do not belong muon track} to true 
hadronic energy in order to determine neutrino total energy.  
\end{itemize}
Analysis done in this thesis is based on two different neutrino samples - contained and uncontained -
which were selected and processed by different algorithms. Muon identification and energy estimation
algorithms for the contained sample are explaned in this chapter, algorithms which were used for 
uncontained sample are explained in separate chapters.

\section{Slicing}
As can be seen on the figure \ref{fig:EVD_full} data recorded for one NuMI spill contains
multple uncorrelated activities. Even though the length of NuMI spill is 10$\mu s$ in the middle
of 550$\mu s$ readout window slicer algorithm is applied to all the hits in the window to protect 
against small possible drift of NuMI spills as well as to make slices for cosmic rays for background
estimation.

The task of the slicer is to group hits, which casually related to each other, in slices. The slicer 
is based on DBSCAN algorithm, one of the most common clustering algorithms,and checks evey pair of hits 
to determine how far are they from each other in space and time. The distance function is not exactly 
spice-time interval but rather modified 
\be
d = \Big(\frac{\Delta t - \frac{\Delta r}{c}}{T_{res}}\Big)^2 + \Big(\frac{\Delta z}{D_z}\Big) + 
\Big(\frac{\Delta v}{D_v}\Big),
\ee 
where $c$ is the speed of light, $\Delta t$ and $\Delta r$ are time and space difference between two 
hits respectively, $T_{res}$ is the timing resolution, $\Delta z$ is the distance between the hits
along the z direction, $\Delta v$ is the distance between hits in x or y direction. As soon as the 
function returns value which is lower than predefined treshold, two hits are put together in one slice.
The first term prefers hits which lie close to a light cone, in other words it is assumed that all
the particles move with the speed of light\footnote{Or energy of the particles is much large than their 
masses.} The second and the third terms penalize hits which are far from each other in space. These
terms help to remove noise hits as their randomly distributed in space and time, and thefore, they 
could be far from the main activity.
