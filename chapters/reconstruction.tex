%%%%%%%%%%%%%%%%%%%%%%%%%%%%%%%%%%%%%%%%%%%%%%%%%%%%%%%%%%%%%%%%%%%%%%%%%%%%%%%%
% reconstruction.tex:
%%%%%%%%%%%%%%%%%%%%%%%%%%%%%%%%%%%%%%%%%%%%%%%%%%%%%%%%%%%%%%%%%%%%%%%%%%%%%%%%
\chapter{Event Reconstruction}
\label{reconstruction_chapter}
%%%%%%%%%%%%%%%%%%%%%%%%%%%%%%%%%%%%%%%%%%%%%%%%%%%%%%%%%%%%%%%%%%%%%%%%%%%%%%%%

%%%%%%%%%%%%%%%%%%%%%%%%%%%%%%%%%%%%%%%%%%%%%%%%%%%%%%%%%%%%%%%%%%%%%%%%%%%%%%%%
In order to perform neutrino oscillation analysis one need to select neutrino events 
and carefully measure its energies. For this task data from the detectors have to be 
reconstructed and $\nu_\mu$ charge current (CC) interactions should be identified. The following
points briefly summarize the reconstruction chain 
\begin{itemize}
\item Slicing. The full 550$\mu s$ readout window contains hits which are associated with
neutrino interaction, cosmis rays activity and noise. The hits need to be separated into
different groups which ideally correspond to different physical processes.
\item Tracking. Group of hits ("slice") from a previous step processed by tracking algorithm
in order to detrmine 3-dimensional trajectories of particles.
\item Calibration. Before energy of the event could be estimated signal from every hit is needed 
to be converted into energy units.
\item Muon identification and Background rejection. $\nu_\mu$ CC interaction produce a muon and, 
thus finding a muon track in a slice indicates that the slice contains information about neutrino.
\item Energy estimation. Templates are used to map muon track length to true muon energy and visible 
hadronic energy\footnote{Total energy of all hits in the slice which do not belong muon track} to true 
hadronic energy in order to determine neutrino total energy.  
\end{itemize}
Analysis done in this thesis is based on two different neutrino samples - contained and uncontained -
which were selected and processed by different algorithms. Muon identification and energy estimation
algorithms for the contained sample are explaned in this chapter, algorithms which were used for 
uncontained sample are explained in separate chapters.

\section{Slicing}
As can be seen on the figure \ref{fig:EVD_full} data recorded for one NuMI spill contains
multple uncorrelated activities. Even though the length of NuMI spill is 10$\mu s$ in the middle
of 550$\mu s$ readout window slicer algorithm is applied to all the hits in the window to protect 
against small possible drift of NuMI spills as well as to make slices for cosmic rays for background
estimation.

The task of the slicer is to group hits, which casually related to each other, in slices. The slicer 
is based on DBSCAN algorithm \cite{DBSCAN}, one of the most common clustering algorithms, and checks 
evey pair of hits to determine how far are they from each other in space and time. The distance 
function is not exactly spice-time interval but rather modified 
\be
d = \Big(\frac{\Delta t - \frac{\Delta r}{c}}{T_{res}}\Big)^2 + \Big(\frac{\Delta z}{D_z}\Big) + 
\Big(\frac{\Delta v}{D_v}\Big),
\ee 
where $c$ is the speed of light, $\Delta t$ and $\Delta r$ are time and space difference between two 
hits respectively, $T_{res}$ is the timing resolution, $\Delta z$ is the distance between the hits
along the z direction, $\Delta v$ is the distance between hits in x or y direction. As soon as the 
function returns value which is lower than predefined treshold, two hits are put together in one slice.
The first term prefers hits which lie close to a light cone, in other words it is assumed that all
the particles move with the speed of light\footnote{Or energy of the particles is much large than their 
masses.} The second and the third terms penalize hits which are far from each other in space. These
terms help to remove noise hits as they are randomly distributed in space and time, and thefore, they 
could be far from the main activity. After all corellated activities have been found and slices are 
created the rest of the hits are put into one additional noise slice.

NOvA slicing algorithms does a good work in separating different neutrino activities in the near detector
and separating cosmic activities from neutrino one in teh far detector. However, hits which were produced
by Michele electrons\footnote{Electron produced in muon decay often called after Louis Michele who
contributed significantly to the physics of charged leptons decay.} or photons follow neutrons capture are
usually not included in the corresponding slice.
\begin{figure}[t]
\includegraphics[width=1.0\textwidth]{figures/Evd_Slicing_FD.pdf}
\centering
\caption{Trigger window in the far detector data. Hits are colored by the time they were recorded.} 
\label{fig:EVD_full}
\end{figure}

\section{Tracking}
As all the activities are separated into corresponding slices the next step in reconstruction chain is 
to find sets of hits which are attributed to the different particles produced in the neutrino interaction 
or cosmic activity. The latter activity usually produced by one particle, while the former one has 
multiple particles. 

NOvA tracking algorithm is based on Kalman filtering \cite{Kalman} and its implementation could be found 
in \cite{Nick}. In short, algorithm is run on two views (X and Y) separately and merge 2D tracks into
3D ones. To construct a 2D track two hits with separation smaller than three planes are choosen as track 
seeds at most downstream location\footnote{Downstream in a sence of incoming neutrino from Fermilab}. 
Algorithm is working upstream and added hits are fit with piecewise linear segments. To add hit or not 
is decided on the basis of how much is the $\chi^2$ is changed for the hit; if less than eight units 
then hit is added. Fitting track with linear segments naturally helps to follow particles trajectory 
with multiple scatterings. Track is considered to be reconstructed if algorithm travels more than 3 planes
and no hits are added. Algorithm is run repetitively until no more tracks could be reconstructed. Majority 
of the particles produced by neutrino interactions are travels dowstream and thus the longest track is 
reconstructed first.

After all 2D tracks are found in both views they need to be merged into 3D tracks. Tracks merging is based
on score function which uses proximity of X- and Y-view tracks start and end points in Z-direction and 
extent of overlap. For each track in X view with Z coordinates $(z_{x1}, z_{x2})$ and track in Y view with
Z coordinates $(z_{y1}, z_{y2})$
\be
\text{score} = \frac{|z_{x1}-z_{y1}| + |z_{x2}-z_{y2}|}{\min(z_{x2},z_{y2}) - \max(z_{x1},z_{y1})}.
\ee
Pair of 2D tracks with the lowest score are corresponded two one particle and they merged into one 3D track.
Merging process continues until there are no 2D tracks to merge, unmatched 2D tracks are also recorded
for possible analysis later.

\section{Calibration}
To convert cell hit arbitrary electronic units (ADC) into meaningful energy units (GeV) a calibration 
process should be carried out. Calibration consists of two parts - attenuation and absolute energy calibration. 
Both parts would be briefly explained in this section.

\subsection{Attenuation correction}
After ADC signal was produced it has to be decided by what factor signal should be multiplied since hit at
the far end of the cell will result in a smaller signal as compared to a signal from a hit in the middle. 
For the obvious reasons hits produced by beam neutrino are no appropriate for the calibration. However,
the abundance of cosmic rays with well understood properties helps with the procedure. Among all reconstructed 
cosmic tracks those are choosen which start and end on detector edges, this criterion is needed to make sure
that tracks are not stopped inside the detector. Additional requirement is that tracks should cross more than
ten planes, thus allowing an accurate estimation of muon path in every cell it crosses. Finally, only those 
hits on tracks are used for callibration which satisfy tri-cell criterion - a hit for callibration should
have both adjacent cells to be hit, see figure \ref{fig:tri-cell}. This conditions lead to a good estimation 
of muon trajectory and hence path length within each cell\footnote{Only for cells which satisfies tri-cell 
criterion.}.
\begin{figure}[h]
\centering
\includegraphics[width=0.9\textwidth]{figures/tri-cell.png}
\caption{Tri-cell creterion.}
{The cell hit is used in calibration only if adjacent cells in the same view are also hit.}
\label{fig:tri-cell}
\end{figure}

The ADC signal is scaled to PE value - estimation for maximal value of electronic signal based on 4 recorded 
ADCs, see chapter \ref{simulation_chapter} for more details. After that for each reference hit, the path length 
and position within the cell ($W$) is estimated and 2D histogram - ratio of PE and path length versus position 
within the cell - is made. Mean ration of PE and path length as function of $W$ is fit by the sum of two exponents
which represent signal trip in two direction along the wave shifting fiber.
\be
y(W) = C + A\Big(e^{\frac{W}{X}} + e^{-\frac{L+W}{X}}\Big),
\ee
where $A$ and $C$ are fit parameters, $X$ is the fiber attenuation length and $L$ is length of the fiber. For
information about additional effects close to cell ends see \cite{calib_technote}. Figure \ref{fig:att_fit} shows 
examples of the fit for far and near detectors.
\begin{figure}[t!]
\begin{subfigure}[t]{0.9\textwidth}
  \centering
  \includegraphics[width=0.9\linewidth]{figures/nd_atten_fit.png}
  \caption{Near Detector}
  \label{fig:att_fit_nd}
\end{subfigure}
\vspace{0.5cm}
\newline
\begin{subfigure}[t]{0.9\textwidth}
  \centering
  \includegraphics[width=0.9\linewidth]{figures/fd_atten_fit.png}
  \caption{Far Detector}
  \label{fig:att_fitt_fd}
\end{subfigure}
\caption{Attenuation fits}
{Examples of attenuation fits in Near (top) and Far (bottom) detectors.}
\label{fig:att_fit}
\end{figure}

For analysis part $W$ of every hit is estimated and PE value of the hit is divided by fit at this posotion to 
get a corrected PE value of the hit. Since attenuation curve determines cell responce to cosmic ray muons, thus 
corrected PE value of the hit shows the relative signal to the cosmic muons. At this point, attenuation is taken
into account but units of corrected PE signal is still arbitrary and absolute energy calibration is still needed. 

\subsection{Absolute energy calibration}
As in the previous step cosmic muons is used at this step too, but rather using middle part of cosmic muon tracks
here is used an end of the track. Thus, tracks which are used for calibration should stop inside the detector. 
The reason for using tracks end is that according to Bethe-Block formula \cite{rpf} muon energy deposition per 
unit length ($\frac{dE}{dx}$) has a well understood minimum near the track end. The minimum deposition is typically
span several meters from the end of the track for different materials and region from 1 meter to 2 meters from
the end of the track is used in NOvA absolute energy calibration procedure \ref{fig:dEdx_vs_length}.
\begin{figure}[t]
\includegraphics[width=1.0\textwidth]{figures/dEdx_vs_length.png}
\centering
\caption{Muon $\frac{dE}{dx}$ per unit length in NOvA.}
{Already calibrated $\frac{dE}{dx}$ is shown as a function of distance to the end of the track.}
\label{fig:dEdx_vs_length}
\end{figure}

\section{Muon Identification}
Neutrino disappearance analysis is heavily relies on correct identification of muon among the byproducts of 
neutrino interaction with detector material. Existence of the muon in the event suggests $\nu_\mu$ CC interaction
and event is passed downstream for further analysis. Currently, NOvA utilizes two fundamentally different 
algorithms to identify muons - k-Nearest Neighbor (kNN) algorithm and convolutional neural network. Extensive 
analysis showed that usage of two algorithms gives better results than any of them separately.

\subsection{ReMID}
ReMID stands for Reconstructed Muon Identification. This algorithm extracts four features from a reconstructed
track and uses them as input to clasification kNN algorithm in order to determine if this particulary track could
be attributed to a muon particle or not. kNN stores training examples in the 4-dimensional (number of dimensions
is equal to number of features extracted from the track) space with additional label specifying the class examples
belong to - muon or non-muon. These training examples were taken from NOvA simulation. Four features from the 
test track are represent a point in this 4-dimensional space and in order to determine the probability the point 
corresponds to a muon track kNN finds its $k$ nearest neighbors\footnote{Regular $L^2$ metric in $R^4$ is used;
$||\vec{r}_1 - \vec{r}_2||^2 = (x_1-x_2)^2 + (y_1-y_2)^2 + (z_1-z_2)^2 + (w_1-w_2)^2$} and calculates the 
fraction of muon nieghbors. This fraction is output of kNN algorithm. Number $k$ is adjustable hyperparameter 
and was choosen to be equal 80 in NOvA. 

Four features which exctracted from the track are track length, $\frac{dE}{dx}$ log-likelihood, scattering 
log-likelihood and non-hadronic plane fraction. $\frac{dE}{dx}$ and scattering log-likelihood are determined with 
the help of probability distributions that shows particle probability to deposit some energy and scatter by some 
angle at some distance from the end of the track, and these distributions are derived from simulation, for more 
information see chapter 6 of \cite{Nick}. The last feature is a ratio of number of planes in the track which are 
free of hadronic shower contamination and total number of planes in the track. 

\subsection{CVN}
CVN stands for Convolutional Visual Network. Up until recently convolutional neural networks were used in industry
only, this type of networks is extremely successful in computer vision tasks such as nubmers recognition on bank 
checks or autonomous driving. It was natural to use this approach in NOvA since detector output looks like nothing
else than a picture. Implementation of CVN in NOvA is described in \cite{CVN}.

In its core CVN is a regular neural network but with tenth of hidden layers and tenth of millions of weights. 
On the higher level though, CVN learns features itself and a human is no longer required to for this work.
An input to CVN is two pictures 100 by 80 pixels around neutrino activity in X and Y views. Similar to ReMID, labels 
for the training pictures are taken from the simulation. However, the biggest difference is that while ReMID checks
one track at a time and does the muon/non-muon classification for track, CVN analyzes the whole activity in the 
slice and does the classification in multiple classes - muon/electron/tay neutrino, CC/RES/DIS/MEC interaction, 
NC interaction and cosmic rays.

Deep convolutional neural network requires tremendous amount of training samples to prevent an overfitting togther 
with regularization  techniques such as dropout layers and weight decay. In order to increase available 
training dataset some pictures are randomly changed. Firstly, Gaussian noise is added to all pixels with a standard 
deviation of 1$\%$, this also reduces reliance on the simulated intensity in each pixel. Secondly, some pictures 
are reflected in the direction perpendicular to neutrino beam. The example of pictures which are fed into CVN can
be seen in figure \ref{fig:cvn_examples}.

\section{Energy Estimation}

