%%%%%%%%%%%%%%%%%%%%%%%%%%%%%%%%%%%%%%%%%%%%%%%%%%%%%%%%%%%%%%%%%%%%%%%%%%%%%%%%
% energy_estimator.tex:
%%%%%%%%%%%%%%%%%%%%%%%%%%%%%%%%%%%%%%%%%%%%%%%%%%%%%%%%%%%%%%%%%%%%%%%%%%%%%%%%
\chapter{Energy Estimator}
\label{energy_estimator_chapter}
%%%%%%%%%%%%%%%%%%%%%%%%%%%%%%%%%%%%%%%%%%%%%%%%%%%%%%%%%%%%%%%%%%%%%%%%%%%%%%%%

This chapter discusses the energy estimation for the selected events with selection
criteria developed in previous chapter. How energy is measured for the contained 
sample one can refer section \ref{energy_est_cont} of event reconstruction chapter.
For the escaping events, energy estimation algorithm is discussed in this chapter.

\section{Different Approaches}
Similar to energy estimation for contained sample events, escaped event is divided
into two parts: muon track (track with the highest ReMId score) and hadronic cluster.
Hadronic cluster is the set of all hits in the slice with removed hits on primary 
track. Preselection cut, which keeps hudronic cluster to at least 3 cells from
the edge of the detector, makes sure that all the activity associated with hadrons is
contained. Thus, hadronic energy of the selected escaping event can be estimated
with the help of the same algorithm as for contained event, but other approach is needed
for the muon track.

\subsection{Quasi-Elastic Formula}
Interaction which constitute a significant fraction of the $\nu_\mu$ CC interactions 
in NOvA is QE interaction. This interaction allows to use an analytic expression for
neutrino energy which is written in terms of obervable variables. For escaping events
observable quantaties are hadronic energy and the primary track angle\footnote{Angle
between muon track and incoming neutrino.}. 

Well known quasi-elastic formula for neutrino energy reconstruction in terms of muon
energy $E_\mu$ and scattering angle %\theta%
\be
E_{\nu} = \frac{E_\mu m_N - \frac{m_\mu^2}{2}}{m_N - E_\mu + p_\mu\cos\theta}
\ee
can be easily rewritten in terms of reconstructed hadronic energy $E_{had}$ and 
scattering angle. With assumptions $E_\mu >> m_\mu$ and that reconstructed hadronic 
energy is kinetic energy of a hadron neutrino interacted with, the previous formula 
tranforms into
\be
E_{\nu} = E_{had}\Big[1 + \sqrt{1 + \frac{4m_N}{E_{had}(1-\cos\theta)}}\Big].
\ee

In practice this approach to neutrino energy reconstruction turned out to be unsatisfactory.
Firstly, figure \ref{fig:trueInt} shows true distribution of neutrino interaction types for
energies in NOvA experiment, for significant fraction of events the formula is not applicable. 
Secondly, reconstructed hadronic energy resolution for lower energy (exactly where QE
events are expected) is worse, see figure 9.18 in \cite{Susan}. Finally, for majority
of muon tracks $\cos\theta$ is close to one which leads to a big uncertainty in 
$\frac{1}{1-\cos\theta}$ term.

\subsection{Lookup Table}
%%%%%%%%%%%%%%%%%%%%%%%%%%%%%%%%%%%%%%%%%%%%%%%%%%%%%%%%%%%%%%%%%%%%%%%%%%%%%%%%
