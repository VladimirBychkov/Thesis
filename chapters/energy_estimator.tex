%%%%%%%%%%%%%%%%%%%%%%%%%%%%%%%%%%%%%%%%%%%%%%%%%%%%%%%%%%%%%%%%%%%%%%%%%%%%%%%%
% energy_estimator.tex:
%%%%%%%%%%%%%%%%%%%%%%%%%%%%%%%%%%%%%%%%%%%%%%%%%%%%%%%%%%%%%%%%%%%%%%%%%%%%%%%%
\chapter{Energy Estimator}
\label{energy_estimator_chapter}
%%%%%%%%%%%%%%%%%%%%%%%%%%%%%%%%%%%%%%%%%%%%%%%%%%%%%%%%%%%%%%%%%%%%%%%%%%%%%%%%

This chapter discusses the energy estimation for the selected events with selection
criteria developed in previous chapter. How energy is measured for the contained 
sample one can refer to section \ref{energy_est_cont} of event reconstruction chapter.
For the escaping events, the energy estimation algorithm is discussed in this chapter.

\section{Different Approaches}
Similar to energy estimation for contained sample events, the escaped event is divided
into two parts: muon track (track with the highest ReMId score) and hadronic cluster.
Hadronic cluster is the set of all hits in the slice with removed hits on primary 
track. Preselection cut, which keeps hudronic cluster to at least 3 cells from
the edge of the detector, makes sure that all the activity associated with hadrons is
contained. Thus, hadronic energy of the selected escaping event can be estimated
with the help of the same algorithm as for contained event, but another approach is needed
for the muon track.

The following subsections briefly describe several approaches taken in attempt to reconstruct
neutrino energy by using either an analytic expression for the neutrino energy or a 
lookup table.

\subsection{Quasi-Elastic Formula}
Interactions which constitute a significant fraction of the $\nu_\mu$ CC interactions 
in NOvA are QE interactions, for these interactions an analytic expression for
neutrino energy, which is written in terms of obervable variables can be used. For escaping events
observable quantaties are hadronic energy and the primary track angle\footnote{Angle
between muon track and incoming neutrino.}. 

The well known quasi-elastic formula for neutrino energy reconstruction in terms of muon
energy $E_\mu$ and scattering angle %\theta%
\be
E_{\nu} = \frac{E_\mu m_N - \frac{m_\mu^2}{2}}{m_N - E_\mu + p_\mu\cos\theta}
\ee
can be easily rewritten in terms of reconstructed hadronic energy $E_{had}$ and 
scattering angle. With assumptions $E_\mu >> m_\mu$ and that reconstructed hadronic 
energy is kinetic energy of a hadron neutrino interacted with, the previous formula 
tranforms into
\be
E_{\nu} = E_{had}\Big[1 + \sqrt{1 + \frac{4m_N}{E_{had}(1-\cos\theta)}}\Big].
\ee
\begin{figure}[!th]
\centering
\includegraphics[width=0.9\textwidth]{figures/neutrino_mode.pdf}
\caption{True Neutrino Interaction Mode.}
{Only $20\%$ of all neutrino interactions which pass Base Cut \ref{evt_sel_sum} are QE, this
fraction even less for the full selection as some CVN QE-like events are not allowed.}
\label{fig:trueInt}
\end{figure}

In practice this approach to neutrino energy reconstruction turned out to be unsatisfactory.
Firstly, figure \ref{fig:trueInt} shows true distribution of neutrino interaction types for
energies in NOvA experiment, for significant fraction of events the formula is not applicable. 
Secondly, reconstructed hadronic energy resolution for lower energy (exactly where QE
events are expected) is worse, see figure 9.18 in \cite{Susan}. Finally, for majority
of muon tracks $\cos\theta$ is close to one which leads to a big uncertainty in 
$\frac{1}{1-\cos\theta}$ term.

\subsection{Lookup Table}
The other approach is to measure hadronic energy of the event together with the scattering
angle of primary track and make a table. Rows and columns of the table are intervals in
reconstructed hadronic energy and scattering angle - 5 bins for hadronic energy from 0 GeV 
to 1 GeV and 5 bins for cosine of scattering angle from 0.5 to 1. For every one of these 25
bins distribution of true energy of neutrino is made and mean is calculated, that mean value
is an entry in the table.

During the test period entry of the table is used as reconstructed energy for the event
with corresponding reconstructed hadronic energy and scattering angle of primary track. 
Results of the approach also turned out to be unsatisfactory independent of number of bins
for reconstructed hadronic energy and scattering angle. Reconstructed energy is mapped to
a narrow region around 2 GeV and no power was left to distinguish between maximal mixing and
non-maximal one. In other words, sensitivity of the experiment was not improved by using 
this energy estimator for escaping events.

\section{Machine Learning Regression Algorithms}
Approaches described in the previous section while being simple and easy interpretable do
not provide good enough improvement in experiment sensitivity. The natural next step in
searching for a good energy estimator for escaping events is to try different machine 
learning regression algorithms such as an artificial neural network, k Nearest Neighbours and
decision trees. All these algorithms are aplicable for regression problem and all of them 
were examined with the same set of input variables.

\subsection{Input Variables}
The first step in construction of muon energy estimator is to determine which variables 
contribute most of the information about true muon energy. The following set is probably
the fullest set of these variables. Extensive search for other variables did not provide 
variables which improved muon energy estimator.
\begin{itemize}
\item Track Length. The length of the primary track contains all the information about muon
energy when track is fully contained in the detector and provides a lower limit for the 
energy otherwise.
\item Track scattering. The cumulutive sum of all deviations from a straight line in degrees
normilized by track length. Muon with a small energy has a shorter track and a lot of scattering 
while propagating through the detector resulting in a bigger track scattering variable. 
This variable descriminates cases with the same amount of contained track but different amount
of escaped part.
\item Track angle. The variable describes the angle between the reconstructed track and 
incoming neutrino\footnote{At the far detector neutrino source is a point source and direction
of incoming neutrinos could be determine.}. 
\item Track Energy Per Hit. Visible energy of the track is divided by number of hits in the 
track. The variable helps to detmine if muon is minimum ionizing particle or not, this has
a strong connection with the muon total energy.
\end{itemize}

\begin{figure}[!th]
\centering
\includegraphics[width=0.9\textwidth]{figures/CorrelationMatrix.pdf}
\caption{Correlation Matrix.}
{Correlations between input variables for BDT algorithms. The algorithm performs better if input
variables are first transform into a Gaussian form and then decorrelated. }
\label{fig:corrMatrix}
\end{figure}

\begin{figure}[!th]
\centering
\includegraphics[width=0.9\textwidth]{figures/InputVarsDG.pdf}
\caption{Input and target variables after decorrelation and gaussinization procedures were performed.}
\label{fig:inputDG}
\end{figure}
During the training period every set of these four variables is supplimented with a true muon
energy in the corresponding event. These variables are not completely independent as 
can be seen in figure \ref{fig:corrMatrix}. It is known that regression algorithms achieve 
better results when input variables are decorrelated, thus decorrelation and gaussinization
procedures were carried out and final distributions before they enter training are shown in figure
\ref{fig:inputDG}.

\subsection{Training}
Training and testing samples contain equal number of events - 101079. This events in addition
to escaping sample selection are subject to one more selection. Namely, events have to be
from the files with subrun number less than 55. This criteria selects events from all
runs with different detector states, so energy estimator during the training process exsposed
to events it could see in the final analysis. Events with subrun equal or more than 55 are used 
for final validation. 

Early tests showed that artificial neural net and k nearest neighbours algorithms have a 
significantly worse performance and these algorithms are not used for final muon energy 
estimator tuning. On the other hand, results for the Boosted Decision Trees (BDT) turned out
to be more promising. The next several paragraph briefly explain how BDT work and what 
hyperparameters are chosen for muon energy estimator.

Regression decision tree is a tree-like graph where data flow from root to leaves through 
several branching points. Root and branching points correspond to questions and if data 
satisfies the question it propogates through one branch, if not then it propagates through
another branch. During the training process tree learns its structure and populates leaves 
with examples from training sample. When a new data point is fed into the tree it propagates
through the tree following the questions in branching points and finally reaches one of the leaves. 
Then regression output of the tree is an average over all training examples in the leaf. 

Unfortunately, the decision tree shows worse performance on average compare to other algorithms 
and tree structure is unstable on training data meaning that small changes leads to drastically 
effect on tree structure. Nevertheless, performance could be improved via ensemble method 
where many different trees are fit and predictions are aggregated across the trees. Ensemble 
sometime called a forest.

One more way to improve performance of decision trees is to use boosting. A function which is 
needed to be learnt, in this case muon energy estimator $E_{muon}$, taken as a weighted sum of 
base functions $f(\vec{x}, \alpha)$ - single tree - often called "weak learners". 
\be
E_{muon}^N(\vec{x}, \vec{\beta}, \vec{\alpha}) = \sum_{n=0}^N \beta_n f(\vec{x}, \alpha_n),
\ee
where $\vec{x}$ is a input vector, $\vec{\beta}$ are weights for different trees and $\vec{\alpha}$
are parameters of the trees. Boosting is a iterative process where each subsequent tree 
$f(\vec{x}, \alpha_{N+1})$ learns in a way to minimize a loss function\footnote{Huber loss function
is used,
$
L(E_{muon}, E_{true}) = \begin{cases}
				\frac{1}{2}(E_{true} - E_{muon})^2 & |E_{true} - E_{muon}| \leq \delta, \\
				\delta(|E_{true} - E_{muon}| - \frac{\delta}{2}) & |E_{true} - E_{muon}| > \delta.
			\end{cases}
$} for $E_{muon}^N(\vec{x}, \vec{\beta}, \vec{\alpha}) + \beta_{N+1}f(\vec{x}, \alpha_{N+1})$.

It is recommended not to use trees with depth deeper than 8, so a depth 5 of was chosen. Another 
hyperparameter for BDT training is the number of weak learners. Ten different options were tried
and the best result was achieved for 800.

\subsection{Performance}
The performance of escaping events energy estimator is evaluated by looking at the RMS of the 
resolution histogram. Resolution is defined as 
\be
\text{Resolution} = \frac{\text{True Muon Energy} - \text{Reco Muon Energy}}{\text{True Muon Energy}}.
\ee
As can be seen in figure \ref{fig:recoMuE_resol}, muon energy resolution is about $\textbf{32\%}$, the 
distribution is not symmetrical and shifted towards positive values meaning that estimator on average 
undershoot the escaping muon energy. Correction could be made to make energy estimator to get right
muon energy on $\textit{average}$ (mean of resolution histogram is zero), however as it shown in the 
next chapter this is not necessary. Figure \ref{fig:recoMuE} illustrates recontructed muon energy together 
with cosmic background. 
\begin{figure}[!th]
\centering
\includegraphics[width=1.0\textwidth]{figures/reco_resol_BDT_mu_Full.pdf}
\caption{Reconstructed Muon Energy Resolution.}
\label{fig:recoMuE_resol}
\end{figure}

\begin{figure}[!th]
\centering
\includegraphics[width=1.0\textwidth]{figures/reco_MuE_BDT_Full.pdf}
\caption{Reconstructed Muon Energy.}
\label{fig:recoMuE}
\end{figure}

For the final fit reconstructed neutrino energy ($E_\nu = E_{\text{had}} + E_\mu$) is used instead 
of just muon energy. Hadronic energy resolution is on the order of $20\%$ which leads to 
$\textbf{23\%}$ reconstructed neutrino energy resolution. Figures \ref{fig:recoE_resol} and 
\ref{fig:recoE} illustrate reconstructed neutrino energy resolution and reconstructed neutrino
energy together with cosmic background. Similar to escaping muon energy, shift to make neutrino
energy correct on $\textit{average}$ is not necessary and it is not improving the experimental sensitivity.
\begin{figure}[!th]
\centering
\includegraphics[width=1.0\textwidth]{figures/reco_resol_BDT_Full.pdf}
\caption{Reconstructed Neutrino Energy Resolution.}
\label{fig:recoE_resol}
\end{figure}

\begin{figure}[!th]
\centering
\includegraphics[width=1.0\textwidth]{figures/reco_E_BDT_Full.pdf}
\caption{Reconstructed Neutrino Energy.}
\label{fig:recoE}
\end{figure}

Another useful plots to look at are the neutrino energy resolution as a function of true and reconstructed
neutrino energy, figures \ref{fig:resol_vs_trueE} and \ref{fig:resol_vs_recoE}. These plots illustrate that 
neutrino energy resolution is stable across the measured neutrino spectrum. Two plots on figure 
\ref{fig:true_vs_reco} show how true energy related to reconstructed energy for muon and neutrino respectively.
\begin{figure}[!th]
\centering
\begin{subfigure}[t]{0.95\textwidth}
  \centering
  \includegraphics[width=1.0\linewidth]{figures/reco_resol_BDT_vs_trueE_Full.pdf}
  \caption{Neutrino Energy Resolution vs True Neutrino Energy.}
  \label{fig:resol_vs_trueE}
\end{subfigure}
\vspace{0.5cm}
\newline
\begin{subfigure}[t]{0.95\textwidth}
  \centering
  \includegraphics[width=1.0\linewidth]{figures/reco_resol_BDT_vs_recoE_Full.pdf}
  \caption{Neutrino Energy Resolution vs Reconstructed Neutrino Energy.}
  \label{fig:resol_vs_recoE}
\end{subfigure}
\caption{ Neutrino Energy Resolution vs Neutrino Energy.}
\label{fig:resol_vs_E}
\end{figure}

\begin{figure}[!th]
\centering
\begin{subfigure}[t]{0.95\textwidth}
  \centering
  \includegraphics[width=1.0\linewidth]{figures/reco_trueMuE_vs_BDT_recoMuE_Full.pdf}
  \caption{True Muon Energy vs Reconstructed Muon Energy.}
  \label{fig:trueMuE_vs_recoMuE}
\end{subfigure}
\vspace{0.5cm}
\newline
\begin{subfigure}[t]{0.95\textwidth}
  \centering
  \includegraphics[width=1.0\linewidth]{figures/reco_trueE_vs_BDT_recoE_Full.pdf}
  \caption{True Neutrino Energy vs Reconstructed Neutrino Energy.}
  \label{fig:trueE_vs_recoE}
\end{subfigure}
\caption{ True Energy vs Reconstructed Energy.}
\label{fig:true_vs_reco}
\end{figure}

\begin{figure}[!th]
\centering
\includegraphics[width=1.0\textwidth]{figures/reco_resol_BDT_vs_len_Full.pdf}
\caption{Neutrino Energy Resolution vs Length of Primary Track.}
\label{fig:resol_vs_len}
\end{figure}

%%%%%%%%%%%%%%%%%%%%%%%%%%%%%%%%%%%%%%%%%%%%%%%%%%%%%%%%%%%%%%%%%%%%%%%%%%%%%%%%
