%%%%%%%%%%%%%%%%%%%%%%%%%%%%%%%%%%%%%%%%%%%%%%%%%%%%%%%%%%%%%%%%%%%%%%%%%%%%%%%%
% conclusion.tex:
%%%%%%%%%%%%%%%%%%%%%%%%%%%%%%%%%%%%%%%%%%%%%%%%%%%%%%%%%%%%%%%%%%%%%%%%%%%%%%%%
\chapter{Conclusion}
\label{conclusion_chapter}
%%%%%%%%%%%%%%%%%%%%%%%%%%%%%%%%%%%%%%%%%%%%%%%%%%%%%%%%%%%%%%%%%%%%%%%%%%%%%%%%

The data analyzed in the present work were taken between 2014 and 2017 in a neutrino mode. The results 
presented in the previous chapter prefer non-maximal mixing with best-fit oscillation parameters 
\be
\sin^2\theta_{23} = 0.504^{+0.079}_{-0.060}, \nn
\ee
\be
\Delta m^2_{32} = 2.45^{+0.11}_{-0.07} \times10^{-3}~ eV^2. \nn
\ee

The selection criteria to select the escaping sample developed in the present thesis allow to select
events which were previously discarded by the analysis software. Despite the marginal improvements in measuring
oscillation parameters using the escaping sample effectively adds \textbf{4\%} more statistics.

The main difficulties with the escaping sample which are cosmic background and event energy estimation were 
addressed. And while energy estimation hardly could be improved more, there are still some space for potential 
improvements in cosmic background rejection. As it can be seen in figure \ref{fig:dataprediction_uncont} the most 
of cosmic background is in the region of 1-2 GeV where experiment is the most sensitive to oscillation 
parameters $\sin^2\theta_{23}$ and $\Delta m^2_{32}$. One of the possible future lines of attack could be a
usage of hit time information to determine the direction of the escaping track\footnote{as of today timing resolution
is not enough, however potentially it could be improved.}. Another possibility could be a deep convolutional neural
net, which was used in $\nu_\mu$ disappearance and $\nu_e$ appearance analyses as well as in this thesis for contained sample,
but trained primarily on escaping events. 

NOvA experiment continues to take data in neutrino and antineutrino mode. Future improvements in analysis techniques,
better understanding of systematics effects and more statistics will help us to understand neutrino sector in greater details 
and, perhaps, will bring more questions to study. Stay tuned!

%%%%%%%%%%%%%%%%%%%%%%%%%%%%%%%%%%%%%%%%%%%%%%%%%%%%%%%%%%%%%%%%%%%%%%%%%%%%%%%%
