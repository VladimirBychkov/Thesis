%%%%%%%%%%%%%%%%%%%%%%%%%%%%%%%%%%%%%%%%%%%%%%%%%%%%%%%%%%%%%%%%%%%%%%%%%%%%%%%%
% abstract.tex: Abstract
%%%%%%%%%%%%%%%%%%%%%%%%%%%%%%%%%%%%%%%%%%%%%%%%%%%%%%%%%%%%%%%%%%%%%%%%%%%%%%%%

The NuMI Off-axis $\nu_e$ Appearance (NOvA) experiment studies neutrino oscillations 
in the NuMI (Neutrinos as the Main Injector) beam. The goal of the experiment is to determine 
mass hierarchy and $\delta_{CP}$, however it is also sensetive to $\sin^2\theta_{23}$ and
$\Delta m_{32}^2$. The present work tries to improve experiment sensitivity to oscillation
parameters by considering neutrino events which are descarded by previous analysis - escaping events.
The biggest issues with the esscaping sample are significant cosmic background and correct
energy estimation. Both issues were addressed and inclusion of escaping sample to the final 
fit led to effective increase of statistics by 4\% with best-fit measurements of 
$\sin^2\theta_{23} = 0.504$ and $\Delta m_{32}^2 = 2.45 \times 10^{-3}$ eV$^2$.

%%%%%%%%%%%%%%%%%%%%%%%%%%%%%%%%%%%%%%%%%%%%%%%%%%%%%%%%%%%%%%%%%%%%%%%%%%%%%%%%
