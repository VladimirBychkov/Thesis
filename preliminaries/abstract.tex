%%%%%%%%%%%%%%%%%%%%%%%%%%%%%%%%%%%%%%%%%%%%%%%%%%%%%%%%%%%%%%%%%%%%%%%%%%%%%%%%
% abstract.tex: Abstract
%%%%%%%%%%%%%%%%%%%%%%%%%%%%%%%%%%%%%%%%%%%%%%%%%%%%%%%%%%%%%%%%%%%%%%%%%%%%%%%%

The NuMI Off-axis $\nu_e$ Appearance (NOvA) experiment studies neutrino oscillations 
in the NuMI (Neutrinos at the Main Injector) beam. The goal of the experiment is to determine 
the mass hierarchy and measure $\delta_{CP}$, however it is also sensitive to $\sin^2\theta_{23}$, and
$\Delta m_{32}^2$. The present work tries to improve experimental sensitivity to the oscillation
parameters by considering neutrino events which are discarded by the previous analysis --- escaping events.
The biggest issues with the escaping sample are significant cosmic background and correct
energy estimation. Both issues were addressed and inclusion of escaping sample to the final 
fit led to an effective increase of statistics by 4\% with best-fit measurements of 
$\sin^2\theta_{23} = 0.504$ and $\Delta m_{32}^2 = 2.45 \times 10^{-3}$ eV$^2$.

%%%%%%%%%%%%%%%%%%%%%%%%%%%%%%%%%%%%%%%%%%%%%%%%%%%%%%%%%%%%%%%%%%%%%%%%%%%%%%%%
